% PlanetMath proposal draft for arXiv:math/9811037
% Source corpora: arXiv math.CT eprints, storage/mo-processed-gpu, storage/math-processed-gpu
\section*{A model for the homotopy theory of homotopy theory}
\subsection*{Source}
\begin{itemize}
  \item arXiv:math/9811037, "A model for the homotopy theory of homotopy theory" (1998-11-06)
  \item StackExchange cross-links via storage/mo-processed-gpu + math-processed-gpu
\end{itemize}
\subsection*{Synopsis}
Rezk describes a category whose objects model homotopy theories, with internal homs showing that functors between homotopy theories again form a homotopy theory. The construction provides a well-behaved setting for "homotopy theory of homotopy theories".
\begin{enumerate}
  \item Builds a category modeling homotopy theories.
  \item Establishes internal hom-objects for functors between homotopy theories.
\end{enumerate}
\subsection*{MathOverflow cues}
\begin{description}
  \item[MO 482463] Rezk models for homotopy theories.
  \item[MO 482489] Internal homs in categories of homotopy theories.
\end{description}
\subsection*{Math.SE anchors}
\begin{description}
  \item[Math.SE 512083] "What is the homotopy theory of homotopy theories?".
  \item[Math.SE 512109] "Calculating internal homs in Rezk's model".
\end{description}
\subsection*{Encyclopedia outline}
\begin{enumerate}
  \item \textbf{Motivation.} Explain why homotopy theories need a model (Math.SE 512083).
  \item \textbf{Model.} Describe Rezk's category and structure (MO 482463).
  \item \textbf{Internal homs.} Summarize existence and properties (MO 482489, Math.SE 512109).
  \item \textbf{Applications.} Note influences on higher category theory.
  \item \textbf{References.} Point to subsequent work.
\end{enumerate}
\subsection*{Action items}
\begin{itemize}
  \item Import the arXiv source to capture definitions and hom constructions.
  \item Link MO 482463/482489 and Math.SE 512083/512109.
  \item Highlight interplay with higher category theory.
\end{itemize}
