% PlanetMath proposal draft for arXiv:2602.05990
% Source corpora: arXiv math.CT eprints, storage/mo-processed-gpu, storage/math-processed-gpu
\section*{Categories graded by group homomorphisms}
\subsection*{Source}
\begin{itemize}
  \item arXiv:2602.05990, "Categories graded by group homomorphisms" (2026-02-05)
  \item StackExchange cross-links via storage/mo-processed-gpu + math-processed-gpu
\end{itemize}
\subsection*{Synopsis}
Jonathan Davies promotes Sözer--Virelizier $\chi$-graded categories to the setting of an arbitrary group homomorphism $\tau\colon G\to H$, simultaneously grading objects and morphisms. A "half-enriched" Yoneda lemma controls representable $\tau$-graded functors, while semisimple $\tau$-graded categories admit block decompositions governed by the image of $\tau$. A pseudofunctor valued description in $\mathbf{Cat}$ packages $\tau$-graded categories as weak actions of $H$ with coherence inherited from $G$.
\begin{enumerate}
  \item Defines $\tau$-graded categories and their morphisms, extending degree constraints to objects.
  \item Establishes a Yoneda lemma tracking homogeneous components of enriched representables.
  \item Proves a structure theorem for semisimple $\tau$-graded categories and compatibly graded functors.
  \item Recasts the theory via pseudofunctors $BH \to \mathbf{Cat}$ endowed with $G$-indexed coherence data.
\end{enumerate}
\subsection*{MathOverflow cues}
\begin{description}
  \item[MO 478120] Clarifies when gradings coming from group homomorphisms produce conservative forgetful functors.
  \item[MO 478134] Discusses how Yoneda arguments adapt to half-enriched graded settings.
\end{description}
\subsection*{Math.SE anchors}
\begin{description}
  \item[Math.SE 507321] "Semisimple gradings indexed by group homomorphisms" -- user examples to cite.
  \item[Math.SE 507466] "Pseudofunctor description of graded categories" -- details on $BH$-actions.
\end{description}
\subsection*{Encyclopedia outline}
\begin{enumerate}
  \item \textbf{Definitions.} Introduce $\tau$-graded categories, homogeneous morphisms, and degree-respecting functors.
  \item \textbf{Yoneda viewpoint.} Explain the half-enriched Yoneda lemma and representability tests (MO 478134).
  \item \textbf{Structure theorems.} Summarise semisimple decompositions and links to Math.SE 507321.
  \item \textbf{Pseudofunctor model.} Detail the $BH \to \mathbf{Cat}$ description and compatibility constraints (Math.SE 507466).
  \item \textbf{Examples.} Work through gradings from abelianization maps and Heisenberg extensions (MO 478120).
\end{enumerate}
\subsection*{Action items}
\begin{itemize}
  \item Import the arXiv source for reference on Yoneda and structure theorems.
  \item Extract MO embeddings 478120/478134 and Math.SE discussions 507321/507466.
  \item Draft PlanetMath entry emphasising graded Yoneda and pseudofunctor reinterpretations.
\end{itemize}
