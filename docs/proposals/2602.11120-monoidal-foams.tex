% PlanetMath proposal draft for arXiv:2602.11120
% Source corpora: arXiv math.CT eprints, storage/mo-processed-gpu, storage/math-processed-gpu
\section*{Monoidal 2-categories from foam evaluation}
\subsection*{Source}
\begin{itemize}
  \item arXiv:2602.11120, ``Monoidal 2-categories from foam evaluation'' (February 2026)
  \item MathOverflow threads on algebras in monoidal 2-categories and foam-based TQFT (IDs 44249, 302385)
  \item Math.StackExchange discussions on Khovanov homology constructions (IDs 414720, 3054088)
\end{itemize}
\subsection*{Synopsis}
Goertz, Marino, and Wedrich provide a blueprint for building locally linear semistrict monoidal $2$-categories out of foam evaluation formulas. Starting from a commutative Frobenius algebra or the $\mathfrak{gl}_N$ foam evaluation of Robert--Wagner, they construct $\Eone$-algebras in graded $\kb$-linear $2$-categories, prove the existence of duals and adjoints, and package the resulting structures into spatially self-dual $2$-multicategories. The framework unifies Bar--Natan cobordisms, $\mathfrak{gl}_N$ foams, and their monoidal refinements appearing in link homology theories.
\begin{enumerate}
  \item Formalizes foam evaluations as the input for semistrict monoidal $2$-categories enriched in graded $\kb$-modules.
  \item Shows that Bar--Natan cobordisms and $\mathfrak{gl}_N$ foams produce $\Alg_{\Eone}(\Cat[\Cat[\kb\grmod]])$ objects generated by 2-dualizable self-dual objects or families of such objects.
  \item Proves that these monoidal $2$-categories are spatial in the sense of Barrett--Meusburger--Schaumann, tying foam data to 3-dimensional pivotality.
  \item Supplies explicit constructions (idempotent completions, grading conventions, traces) needed for link homology computations.
  \item Suggests extensions to other diagrammatic categorifications and foamy models of 2D TQFTs.
\end{enumerate}
\subsection*{MathOverflow cues}
\begin{description}
  \item[MO 44249] Explains how algebras in monoidal $2$-categories behave, a language used in the paper for locating foam-based $\Eone$-algebras.
  \item[MO 302385] Discusses foam interpretations of instanton/Khovanov-type invariants, framing the motivation for encoding foams in higher categorical structures.
\end{description}
\subsection*{Math.SE anchors}
\begin{description}
  \item[Math.SE 414720] Questions linear maps in Khovanov homology, illustrating why categorical control over foam operations is desirable.
  \item[Math.SE 3054088] Relates spanning trees of Tait graphs to Khovanov data; useful for examples connecting foams, Hochschild-type gradings, and monoidal traces.
\end{description}
\subsection*{Encyclopedia outline}
\begin{enumerate}
  \item Foam evaluation data and the construction of semistrict monoidal $2$-categories.
  \item Case studies: Bar--Natan cobordisms and $\mathfrak{gl}_N$ foams.
  \item Spatial duality structures and consequences for link homology.
  \item Relationship with $\Eone$-algebras and enriched $2$-categorical language.
  \item Outlook toward diagrammatic categorification and higher TQFT models.
\end{enumerate}
\subsection*{Action items}
\begin{itemize}
  \item Summarize the categorical construction (objects, 1-morphisms, 2-morphisms, tensor structure) and explain how foam relations guarantee associativity/duality.
  \item Document the examples (Bar--Natan over graded Frobenius algebras, $\Foamssmonbicat{N}$) and reference links to PlanetMath entries on Khovanov/Bar--Natan invariants.
  \item Highlight spatial duality notions and cite Barrett--Meusburger--Schaumann for background.
  \item Provide guidance on how these constructions feed into generalized trace or state-sum invariants for link homology.
\end{itemize}
