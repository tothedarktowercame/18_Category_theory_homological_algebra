% PlanetMath proposal draft for arXiv:math/9910006
% Source corpora: arXiv math.CT eprints, storage/mo-processed-gpu, storage/math-processed-gpu
\section*{The syntax of coherence}
\subsection*{Source}
\begin{itemize}
  \item arXiv:math/9910006, "The Syntax of Coherence" (1999-10-01)
  \item StackExchange cross-links via storage/mo-processed-gpu + math-processed-gpu
\end{itemize}
\subsection*{Synopsis}
Noson Yanofsky presents 2-theories (syntactic descriptions of structured categories) to tackle coherence via a 2D generalization of Lawvere functorial semantics. Quasi-Yoneda lemma viewpoints show coherence results as statements about 2-theory morphisms, yielding quasi-adjoints replacing structures. A two-dimensional Kronecker product generates new coherence laws.
\begin{enumerate}
  \item Introduces 2-theories for structured categories.
  \item Describes quasi-Yoneda lemma consequences.
  \item Constructs quasi-adjoints and 2D Kronecker products.
\end{enumerate}
\subsection*{MathOverflow cues}
\begin{description}
  \item[MO 484879] 2-theories and coherence.
  \item[MO 484906] Quasi-adjoints in 2-theory morphisms.
\end{description}
\subsection*{Math.SE anchors}
\begin{description}
  \item[Math.SE 514681] "Understanding 2-theories".
  \item[Math.SE 514708] "Kronecker product in higher coherence".
\end{description}
\subsection*{Encyclopedia outline}
\begin{enumerate}
  \item \textbf{2-theories.} Define syntax and semantics (Math.SE 514681).
  \item \textbf{Coherence via quasi-Yoneda.} Present main idea (MO 484879).
  \item \textbf{Quasi-adjoints.} Explain replacements of structure (MO 484906).
  \item \textbf{2D Kronecker product.} Summarize construction (Math.SE 514708).
  \item \textbf{Applications.} Mention generating new coherence laws.
\end{enumerate}
\subsection*{Action items}
\begin{itemize}
  \item Import arXiv explanations of 2-theories, quasi-adjoints, and Kronecker product.
  \item Link MO 484879/484906 and Math.SE 514681/514708.
  \item Emphasize syntactic approach to coherence.
\end{itemize}
