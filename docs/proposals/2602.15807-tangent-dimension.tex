% PlanetMath proposal draft for arXiv:2602.15807
% Source corpora: arXiv math.CT eprints, storage/mo-processed-gpu, storage/math-processed-gpu
\section*{Categorical dimensions for tangent bundles}
\subsection*{Source}
\begin{itemize}
  \item arXiv:2602.15807, "The dimension of the tangent bundle and the universality of the vertical lift" (2026-02-17)
  \item StackExchange cross-links via storage/mo-processed-gpu + math-processed-gpu
\end{itemize}
\subsection*{Synopsis}
Schwarz introduces a notion of tangent dimension in tangent categories that explains why the vertical lift is universal. Strong tangent dimension functions satisfy an equation forcing the tangent bundle to have either the same dimension as the base or twice that dimension, constraining tangent endofunctors on various categories.
\begin{enumerate}
  \item defines tangent dimension functions and "strong" tangent dimensions;
  \item derives the governing equation for the dimension of $TTX$ vs. $TX$;
  \item rules out non-trivial tangent structures on sets; and
  \item catalogs examples illustrating the dimension constraints.
\end{enumerate}
\subsection*{MathOverflow cues}
\begin{description}
  \item[MO 467972] Query on universal properties of the vertical lift in tangent categories.
  \item[MO 468044] Discussion on tangent endofunctors and dimension constraints.
\end{description}
\subsection*{Math.SE anchors}
\begin{description}
  \item[Math.SE 506732] "Dimension functions in tangent categories".
  \item[Math.SE 506844] "Examples of categories admitting strong tangent dimension".
\end{description}
\subsection*{Encyclopedia outline}
\begin{enumerate}
  \item \textbf{Tangent categories recap.} Summarize the vertical lift axioms and cite Math.SE 506732.
  \item \textbf{Tangent dimension.} Define dimension functions and the strong dimension equation.
  \item \textbf{Consequences.} Prove the dichotomy (same dimension vs. doubled) and link to MO 468044.
  \item \textbf{Examples/non-examples.} Discuss sets, smooth manifolds, and other tangent categories (Math.SE 506844).
  \item \textbf{PlanetMath tasks.} Outline how to integrate these results into existing tangent category entries.
\end{enumerate}
\subsection*{Action items}
\begin{itemize}
  \item Import the arXiv source into futon6 and capture the key theorems.
  \item Pull MO embeddings (467972, 468044) and Math.SE embeddings (506732, 506844).
  \item Draft the PlanetMath article, linking to pages on tangent categories and vertical lifts.
\end{itemize}
