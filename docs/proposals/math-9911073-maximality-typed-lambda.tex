% PlanetMath proposal draft for arXiv:math/9911073
% Source corpora: arXiv math.CT eprints, storage/mo-processed-gpu + math-processed-gpu
\section*{The maximality of the typed lambda calculus and of cartesian closed categories}
\subsection*{Source}
\begin{itemize}
  \item arXiv:math/9911073, "The Maximality of the Typed Lambda Calculus and of Cartesian Closed Categories" (1999-11-11)
  \item StackExchange cross-links via storage/mo-processed-gpu + math-processed-gpu
\end{itemize}
\subsection*{Synopsis}
Dosen and Petric, using a Boehm-type theorem, prove that any cartesian closed category satisfying an equality not valid in the free CCC must be a preorder; similarly for typed lambda calculus (with or without products). They provide new proofs of Statman/Simpson results.
\begin{enumerate}
  \item Presents Boehm-style theorem for typed lambda calculus variants.
  \item Concludes that extra equalities collapse CCCs to preorders.
  \item Supplies new proofs of maximality results.
\end{enumerate}
\subsection*{MathOverflow cues}
\begin{description}
  \item[MO 485194] Maximality of cartesian closed categories.
  \item[MO 485221] Boehm theorems for typed lambda calculus.
\end{description}
\subsection*{Math.SE anchors}
\begin{description}
  \item[Math.SE 515025] "When extra CCC equations force preorders".
  \item[Math.SE 515051] "Boehm-like results for typed lambda calculus".
\end{description}
\subsection*{Encyclopedia outline}
\begin{enumerate}
  \item \textbf{Free CCCs.} Review structure and equalities (Math.SE 515025).
  \item \textbf{Boehm theorem analogue.} Summarize main lemma (MO 485221).
  \item \textbf{Maximality.} Explain collapse to preorders (MO 485194).
  \item \textbf{Lambda calculus implications.} Present with/without products.
  \item \textbf{Historical context.} Mention Statman and Simpson.
\end{enumerate}
\subsection*{Action items}
\begin{itemize}
  \item Import arXiv proofs of Boehm-style theorem and maximality.
  \item Link MO 485194/485221 and Math.SE 515025/515051.
  \item Highlight consequences for typed lambda calculus and CCCs.
\end{itemize}
