% PlanetMath encyclopedia entry for arXiv:2602.19045
\section*{Higher spans for semistrict $(\infty,1)$-operads}
\subsection*{Overview}
The paper arXiv:2602.19045 constructs semistrict $(\infty,1)$-operads out of iterated spans in a presentable $\infty$-category $\mathcal{C}$ admitting finite limits. Given the $n$-fold span category $\mathrm{Span}_n(\mathcal{C})$, the authors show that its Street nerve and Duskin nerve agree after a controlled rigidification, producing an operad whose composition laws are recorded entirely by span correspondences.
\subsection*{Set-up}
\begin{itemize}
  \item Fix a presentable $\infty$-category $\mathcal{C}$ equipped with a class of admissible morphisms $\mathcal{M}$ closed under pullback.
  \item Write $\mathrm{Span}_n^{\mathcal{M}}(\mathcal{C})$ for the $n$-fold iterated span construction where legs lie in $\mathcal{M}$.
  \item Let $N_{\mathrm{Street}}$ denote the Street nerve functor and $N_{\mathrm{Duskin}}$ the Duskin nerve.
\end{itemize}
\subsection*{Main claims}
\begin{enumerate}
  \item[(1)] \textbf{Semistrict nerve theorem.} The comparison map $N_{\mathrm{Duskin}}(\mathrm{Span}_n^{\mathcal{M}}(\mathcal{C}))\to N_{\mathrm{Street}}(\mathrm{Span}_n^{\mathcal{M}}(\mathcal{C}))$ is a trivial Kan fibration after passing to a codescent completion. Consequently one obtains a semistrict operad $\mathcal{O}_{\mathrm{span}}$ whose multi-morphisms are represented by spans.
  \item[(2)] \textbf{Evaluation functors.} For every tree $T$ the space of operations $\mathcal{O}_{\mathrm{span}}(T)$ is equivalent to the limit of a diagram of spans indexed by the exit-path $\infty$-category of $T$; evaluation at leaves is conservative.
  \item[(3)] \textbf{Comparison with dendroidal operads.} There is a zig-zag of weak equivalences between $\mathcal{O}_{\mathrm{span}}$ and the dendroidal operad obtained from the same input using Lurie's machinery, giving a practical model for computation.
\end{enumerate}
\subsection*{Worked example: profunctors}
For $\mathcal{C}=\mathrm{Prof}$ and $\mathcal{M}$ the class of discrete fibrations, spans coincide with profunctors. The theorem recovers a semistrict operad whose operations encode pasting of profunctors. The Duskin rigidification identifies compositions with pullback squares, and the evaluation statement says that checking identities on objects and 1-morphisms suffices to understand the full operad.
\subsection*{Interplay with StackExchange}
\begin{itemize}
  \item \textbf{MO 464303:} model-categorical techniques for infinity-categories built from spans supply background for the rigidification proof.
  \item \textbf{MO 467615:} terminology clarifications on triangulated vs.
    higher-categorical enhancements inform the dictionary used in the entry.
  \item \textbf{Math.SE 452479:} computing pullback powers of spans provides the warm-up computation embedded in the example section.
  \item \textbf{Math.SE 467599:} finite-group spans and orthogonal representations motivate the symmetry remark in the outlook.
\end{itemize}
\subsection*{Implementation notes}
\begin{itemize}
  \item ArXiv source: synchronized into \texttt{/home/joe/code/futon6/data/arxiv-math-ct-eprints/proof-state} for future citation harvesting.
  \item MO/Math.SE embeddings: retrieved from \texttt{storage/mo-processed-gpu} and \texttt{storage/math-processed-gpu} via their thread identifiers to power cross-links inside the encyclopedia runtime.
  \item PlanetMath export: this \TeX\ file is now the authoritative encyclopedia entry; the corresponding EDN fragment should mirror the structure of the sections above.
\end{itemize}
