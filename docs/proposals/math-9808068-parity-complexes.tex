% PlanetMath proposal draft for arXiv:math/9808068
% Source corpora: arXiv math.CT eprints, storage/mo-processed-gpu, storage/math-processed-gpu
\section*{On Parity Complexes and Non-abelian Cohomology}
\subsection*{Source}
\begin{itemize}
  \item arXiv:math/9808068, "On Parity Complexes and Non-abelian Cohomology" (1998-08-15)
  \item StackExchange cross-links via storage/mo-processed-gpu + math-processed-gpu
\end{itemize}
\subsection*{Synopsis}
Ionescu introduces parity (quasi)complexes to encode associativity, commutativity, and monoidality constraints in quasimonoidal categories. Categorifying group cohomology, he interprets 1-cochains as functors, 2-cocycles as monoidal structures, 3-cocycles as associators, yielding non-abelian cohomology descriptions and identifying broader commutativity constraints linked to coboundary Hopf algebras.
\begin{enumerate}
  \item Defines parity complexes and relates them to non-abelian cohomology.
  \item Categorifies group cohomology to quasimonoidal categories via parity complexes.
  \item Identifies larger families of commutativity constraints associated to coboundary Hopf algebras.
\end{enumerate}
\subsection*{MathOverflow cues}
\begin{description}
  \item[MO 481475] Parity complexes in quasimonoidal categories.
  \item[MO 481502] Non-abelian cohomology interpretations of associators.
\end{description}
\subsection*{Math.SE anchors}
\begin{description}
  \item[Math.SE 511086] "Categorifying group cohomology".
  \item[Math.SE 511112] "Coboundary Hopf algebras and commutativity constraints".
\end{description}
\subsection*{Encyclopedia outline}
\begin{enumerate}
  \item \textbf{Parity complexes.} Define objects and chain conditions (MO 481475).
  \item \textbf{Cohomology link.} Explain categorified cohomology (Math.SE 511086).
  \item \textbf{Commutativity.} Discuss generalized constraints (MO 481502, Math.SE 511112).
  \item \textbf{Applications.} Mention quasi-extensions and monoidal structures.
  \item \textbf{Examples.} Provide explicit parity complex computations.
\end{enumerate}
\subsection*{Action items}
\begin{itemize}
  \item Import the arXiv source for definitions and examples.
  \item Link MO 481475/481502 and Math.SE 511086/511112.
  \item Emphasize cohomological interpretation in PlanetMath entry.
\end{itemize}
