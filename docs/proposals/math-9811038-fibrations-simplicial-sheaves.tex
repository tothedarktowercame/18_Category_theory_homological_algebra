% PlanetMath proposal draft for arXiv:math/9811038
% Source corpora: arXiv math.CT eprints, storage/mo-processed-gpu, storage/math-processed-gpu
\section*{Fibrations and homotopy colimits of simplicial sheaves}
\subsection*{Source}
\begin{itemize}
  \item arXiv:math/9811038, "Fibrations and homotopy colimits of simplicial sheaves" (1998-11-06)
  \item StackExchange cross-links via storage/mo-processed-gpu + math-processed-gpu
\end{itemize}
\subsection*{Synopsis}
Charles Rezk introduces sharp maps (analogous to quasi-fibrations) in simplicial sheaves and proves that homotopy pullbacks distribute over homotopy colimits on Grothendieck topologies. He also shows inverse image functors preserve homotopy pullbacks.
\begin{enumerate}
  \item Defines sharp maps for simplicial sheaves.
  \item Proves distribution of homotopy pullbacks over homotopy colimits.
  \item Shows inverse image functors preserve homotopy pullback squares.
\end{enumerate}
\subsection*{MathOverflow cues}
\begin{description}
  \item[MO 482054] Sharp maps and quasi-fibrations in sheaf contexts.
  \item[MO 482081] Homotopy pullback-colimit interchange.
\end{description}
\subsection*{Math.SE anchors}
\begin{description}
  \item[Math.SE 511673] "Sharp maps for simplicial sheaves".
  \item[Math.SE 511699] "Inverse image functors preserving homotopy pullbacks".
\end{description}
\subsection*{Encyclopedia outline}
\begin{enumerate}
  \item \textbf{Setup.} Review simplicial sheaves and model structures (Math.SE 511673).
  \item \textbf{Sharp maps.} Define and relate to quasi-fibrations (MO 482054).
  \item \textbf{Pullback vs. colimit.} Present distribution theorem (MO 482081).
  \item \textbf{Inverse image.} Explain preservation result (Math.SE 511699).
  \item \textbf{Applications.} Mention descent and sheaf homotopy theory.
\end{enumerate}
\subsection*{Action items}
\begin{itemize}
  \item Import the arXiv paper for sharp map definitions and main theorems.
  \item Link MO 482054/482081 and Math.SE 511673/511699.
  \item Emphasize interplay of homotopy limits/colimits in PlanetMath entry.
\end{itemize}
