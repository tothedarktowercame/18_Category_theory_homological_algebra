% PlanetMath proposal draft for arXiv:math/9908006
% Source corpora: arXiv math.CT eprints, storage/mo-processed-gpu, storage/math-processed-gpu
\section*{Topological gravity in dimensions two and four}
\subsection*{Source}
\begin{itemize}
  \item arXiv:math/9908006, "Topological gravity in dimensions two and four" (1999-08-01)
  \item StackExchange cross-links via storage/mo-processed-gpu + math-processed-gpu
\end{itemize}
\subsection*{Synopsis}
Jack Morava generalizes physicists' work on 2D gravity to 4D using analogues of Segal's category for conformal field theory, framing topological gravity in categorical terms.
\begin{enumerate}
  \item Reviews 2D topological gravity construction.
  \item Extends framework to four dimensions via Segal-type categories.
\end{enumerate}
\subsection*{MathOverflow cues}
\begin{description}
  \item[MO 485071] Segal categories and topological gravity.
  \item[MO 485098] Extending 2D gravity frameworks.
\end{description}
\subsection*{Math.SE anchors}
\begin{description}
  \item[Math.SE 514895] "Topological gravity categories".
  \item[Math.SE 514921] "Analogues of Segal's category".
\end{description}
\subsection*{Encyclopedia outline}
\begin{enumerate}
  \item \textbf{Segal's category.} Review conformal field theory perspective (Math.SE 514921).
  \item \textbf{2D gravity.} Summarize existing work (MO 485098).
  \item \textbf{4D extension.} Explain Morava's proposal (MO 485071, Math.SE 514895).
  \item \textbf{Implications.} Discuss topological gravity structures.
  \item \textbf{Examples.} Provide simple categorical constructions.
\end{enumerate}
\subsection*{Action items}
\begin{itemize}
  \item Import arXiv discussion on categorical topological gravity.
  \item Link MO 485071/485098 and Math.SE 514895/514921.
  \item Highlight generalization from 2D to 4D.
\end{itemize}
