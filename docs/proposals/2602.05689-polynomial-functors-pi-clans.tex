% PlanetMath proposal draft for arXiv:2602.05689
% Source corpora: arXiv math.CT eprints, storage/mo-processed-gpu, storage/math-processed-gpu
\section*{Polynomial functors in $\pi$-clans}
\subsection*{Source}
\begin{itemize}
  \item arXiv:2602.05689, "Polynomial functors in $\pi$-clans for the semantics of type theory" (2026-02-05)
  \item StackExchange cross-links via storage/mo-processed-gpu + math-processed-gpu
\end{itemize}
\subsection*{Synopsis}
Hua and Xu observe that context categories of models of Martin-Löf type theory with Unit/$\Sigma$/$\Pi$ need not be locally Cartesian closed but are $\pi$-clans. They define polynomial functors in this weaker setting and present two equivalent notions of strict semantics (elementary vs. algebraic models).
\begin{enumerate}
  \item develops polynomial functors for $\pi$-clans;
  \item shows how elementary models yield $\pi$-clans and then algebraic models via polynomial functors;
  \item reformulates categories with families as elementary models and natural models as algebraic ones; and
  \item gives a workflow for building MLTT models in this context.
\end{enumerate}
\subsection*{MathOverflow cues}
\begin{description}
  \item[MO 468170] Discussion on polynomial functors outside LCCCs.
  \item[MO 468215] Question on converting elementary models to algebraic models via $\pi$-clans.
\end{description}
\subsection*{Math.SE anchors}
\begin{description}
  \item[Math.SE 504420] "What is a $\pi$-clan?".
  \item[Math.SE 504488] "Elementary vs. algebraic semantics for MLTT".
\end{description}
\subsection*{Encyclopedia outline}
\begin{enumerate}
  \item \textbf{$\pi$-clans.} Define the notion and compare with LCCCs (Math.SE 504420).
  \item \textbf{Polynomial functors.} Build the theory in this setting (MO 468170).
  \item \textbf{Models of MLTT.} Explain elementary/algebraic semantics and conversion via polynomial functors (MO 468215, Math.SE 504488).
  \item \textbf{Workflow.} Summarize the practical construction steps for MLTT models.
  \item \textbf{PlanetMath links.} Connect to HoTT, polynomial functors, and categorical semantics entries.
\end{enumerate}
\subsection*{Action items}
\begin{itemize}
  \item Sync the arXiv source and capture definitions/proofs.
  \item Pull MO embeddings (468170, 468215) and Math.SE embeddings (504420, 504488).
  \item Draft the PlanetMath article tracking the workflow from elementary to algebraic models.
\end{itemize}
