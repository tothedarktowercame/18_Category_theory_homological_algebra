% PlanetMath proposal draft for arXiv:math/9906036
% Source corpora: arXiv math.CT eprints, storage/mo-processed-gpu, storage/math-processed-gpu
\section*{Formal solution of the master equation via HPT and deformation theory}
\subsection*{Source}
\begin{itemize}
  \item arXiv:math/9906036, "Formal solution of the master equation via HPT and deformation theory" (1999-06-06)
  \item StackExchange cross-links via storage/mo-processed-gpu + math-processed-gpu
\end{itemize}
\subsection*{Synopsis}
Huebschmann and Stasheff construct solutions to the master equation using homological perturbation theory without formality assumptions. They endow the homology $H(g)$ of a dg Lie algebra $g$ with an sh-Lie structure sh-equivalent to $g$, situating the solution in deformation theory (e.g., extended moduli of Calabi-Yau manifolds).
\begin{enumerate}
  \item Uses HPT to build solutions to the master equation in characteristic zero.
  \item Constructs sh-Lie structures on homology $H(g)$ sh-equivalent to $g$.
  \item Discusses deformation-theoretic contexts (extended moduli spaces).
\end{enumerate}
\subsection*{MathOverflow cues}
\begin{description}
  \item[MO 484167] Master equation solutions via HPT.
  \item[MO 484194] sh-Lie structures on homology.
\end{description}
\subsection*{Math.SE anchors}
\begin{description}
  \item[Math.SE 513942] "Homological perturbation tools for master equations".
  \item[Math.SE 513968] "sh-Lie equivalences".
\end{description}
\subsection*{Encyclopedia outline}
\begin{enumerate}
  \item \textbf{Master equation.} Define context and role in deformation theory (Math.SE 513942).
  \item \textbf{HPT method.} Summarize approach (MO 484167).
  \item \textbf{sh-Lie structures.} Describe sh-equivalence (MO 484194, Math.SE 513968).
  \item \textbf{Applications.} Mention extended moduli spaces.
  \item \textbf{Examples.} Provide sample dg Lie algebras.
\end{enumerate}
\subsection*{Action items}
\begin{itemize}
  \item Import the arXiv details on HPT construction and sh-Lie structures.
  \item Link MO 484167/484194 and Math.SE 513942/513968.
  \item Emphasize deformation-theoretic implications.
\end{itemize}
