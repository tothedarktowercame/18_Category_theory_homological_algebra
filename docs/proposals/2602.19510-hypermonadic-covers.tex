% PlanetMath proposal draft for arXiv:2602.19510
% Source corpora: arXiv math.CT eprints, storage/mo-processed-gpu, storage/math-processed-gpu
\section*{Hypermonadic covers in higher topos theory}
\subsection*{Source}
\begin{itemize}
  \item arXiv:2602.19510, "Hypermonadic covers in higher topos theory" (2026-02-22)
  \item StackExchange cross-links via storage/mo-processed-gpu + math-processed-gpu
\end{itemize}
\subsection*{Synopsis}
Hypermonadic covers refine ordinary hypercovers by threading explicit monadic descent data through a cohesive $\infty$-topos. The paper emphasizes descent-theoretic comparisons rather than span combinatorics, so this proposal complements the span-heavy condensed entries by focusing on \emph{monadic towers} and their connectivity theorems. The authors:
\begin{enumerate}
  \item build a tower resolving a cohesive topos $\mathcal{X}$ via successive monadic approximations and compare it with hypercovers;
  \item show that the resulting hypermonadic nerve detects the same $\infty$-connectivity as Grothendieck topologies;
  \item describe how condensed symmetry acts on hypermonadic towers, producing spectral descent sequences; and
  \item work out examples for condensed abelian groups, citing Math.SE 500112/502640 for background.
\end{enumerate}
\subsection*{MathOverflow cues}
\begin{description}
  \item[MO 466823] Hypercover descent question comparing Čech vs. cohomological connectivity; we will cite it in the background section.
  \item[MO 467143] Cohesive topos discussion on monadic descent data that mirrors the spectral sequence argument.
\end{description}
\subsection*{Math.SE anchors}
\begin{description}
  \item[Math.SE 500112] "How do hypercovers appear in higher topoi?"
  \item[Math.SE 502640] "Monad descent in $(\infty,1)$-categories".
\end{description}
\subsection*{Encyclopedia outline}
\begin{enumerate}
  \item \textbf{Cohesive hypercovers recap.} Review hypercovers in cohesive $\infty$-topoi, summarize Math.SE 500112, and explain the gaps a monadic refinement fills.
  \item \textbf{Hypermonadic tower.} Define the monadic approximation tower, illustrate how each stage functions, and restate the comparison with classical hypercovers (Math.SE 502640 + MO 466823).
  \item \textbf{Connectivity theorem.} Present the equivalence between hypermonadic nerves and Grothendieck topologies, including the spectral sequence argument referenced in MO 467143.
  \item \textbf{Condensed symmetry action.} Describe how condensed groups act on hypermonadic towers, linking to concrete condensed abelian examples.
  \item \textbf{Examples + PlanetMath tasks.} Work out the condensed abelian group case explicitly, list the data pipeline (arXiv source + storage/mo/math embeddings), and note how the result feeds into future entries on descent.
\end{enumerate}
\subsection*{Action items}
\begin{itemize}
  \item Pull the arXiv source into futon6 for quoting the hypermonadic tower construction.
  \item Fetch MO embeddings (466823, 467143) and Math.SE embeddings (500112, 502640) to reuse the descent questions verbatim.
  \item Produce the PlanetMath draft using the expanded outline and link to related hypercover entries.
\end{itemize}
