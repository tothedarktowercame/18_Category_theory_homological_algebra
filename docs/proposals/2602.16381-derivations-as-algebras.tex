% PlanetMath proposal draft for arXiv:2602.16381
% Source corpora: arXiv math.CT eprints, storage/mo-processed-gpu, storage/math-processed-gpu
\section*{Derivations as algebras in differential categories}
\subsection*{Source}
\begin{itemize}
  \item arXiv:2602.16381, "Derivations as Algebras" (2026-02-18)
  \item StackExchange cross-links via storage/mo-processed-gpu + math-processed-gpu
\end{itemize}
\subsection*{Synopsis}
Pacaud Lemay and Sava show that the differential modality of a differential category lifts to a monad on its arrow category whose algebras are derivations. In the presence of finite biproducts, the arrow category inherits a differential modality, so derivations form both a tangent category and (for free algebras) a cartesian differential category.
\begin{enumerate}
  \item lifts the differential modality to the arrow category and identifies its algebras with derivations;
  \item proves the arrow category of a differential category is again differential when biproducts exist;
  \item shows derivations form a tangent category; and
  \item analyzes derivations on free algebras as a cartesian differential category.
\end{enumerate}
\subsection*{MathOverflow cues}
\begin{description}
  \item[MO 468204] Question about lifting differential modalities to arrow categories.
  \item[MO 468250] Thread on when derivations carry tangent/cartan differential structure.
\end{description}
\subsection*{Math.SE anchors}
\begin{description}
  \item[Math.SE 507280] "Differential categories and derivations".
  \item[Math.SE 507362] "Why do arrow categories inherit differential structure?"
\end{description}
\subsection*{Encyclopedia outline}
\begin{enumerate}
  \item \textbf{Differential categories recap.} Summarize the differential modality and cite Math.SE 507280.
  \item \textbf{Arrow-category monad.} Describe the lifted modality, referencing MO 468204.
  \item \textbf{Differential/tangent structure.} Explain how the arrow category becomes differential and tangent (Math.SE 507362, MO 468250).
  \item \textbf{Free algebra case.} Detail the cartesian differential structure on derivations of free algebras.
  \item \textbf{PlanetMath integration.} Plan cross-links to tangent categories, linear logic, and derivation entries.
\end{enumerate}
\subsection*{Action items}
\begin{itemize}
  \item Pull the arXiv source for precise statements/proofs.
  \item Fetch MO embeddings (468204, 468250) and Math.SE embeddings (507280, 507362).
  \item Draft the PlanetMath entry following the outline and reference existing derivation pages.
\end{itemize}
