% PlanetMath proposal draft for arXiv:math/9802028
% Source corpora: arXiv math.CT eprints, storage/mo-processed-gpu, storage/math-processed-gpu
\section*{Cross Product Bialgebras -- Part I}
\subsection*{Source}
\begin{itemize}
  \item arXiv:math/9802028, "Cross Product Bialgebras -- Part I" (1998-02-05)
  \item StackExchange cross-links via storage/mo-processed-gpu + math-processed-gpu
\end{itemize}
\subsection*{Synopsis}
Bespalov and Drabant present a universal characterization of cross product bialgebras (without cocycles) using projections/injections and Hopf data. The framework encompasses biproducts, double cross products, bicross products, and yields new co-cycle-free constructions inside braided monoidal categories, especially Hopf bimodules.
\begin{enumerate}
  \item Develops a universal characterization in terms of projections/injections.
  \item Provides a (co-)modular description via Hopf data.
  \item Applies to braided categories of Hopf bimodules, recovering known examples and generating new ones.
\end{enumerate}
\subsection*{MathOverflow cues}
\begin{description}
  \item[MO 480914] Hopf data encoding cross product bialgebras.
  \item[MO 480941] Double/bicross product comparisons in braided settings.
\end{description}
\subsection*{Math.SE anchors}
\begin{description}
  \item[Math.SE 510528] "Universal properties of cross product bialgebras".
  \item[Math.SE 510556] "Hopf bimodules as braided categories".
\end{description}
\subsection*{Encyclopedia outline}
\begin{enumerate}
  \item \textbf{Background.} Review biproduct/double cross/bicross constructions (Math.SE 510528).
  \item \textbf{Universal characterization.} Summarize projection/injection criteria (MO 480914).
  \item \textbf{Hopf data.} Explain (co-)modular descriptions (Math.SE 510556).
  \item \textbf{Applications.} Cover Hopf bimodule and Majid double biproduct cases (MO 480941).
  \item \textbf{Examples.} Provide explicit braided category computations.
\end{enumerate}
\subsection*{Action items}
\begin{itemize}
  \item Import the arXiv source to capture universal theorems and examples.
  \item Gather MO 480914/480941 and Math.SE 510528/510556 references.
  \item Draft PlanetMath entry highlighting Hopf data formulations.
\end{itemize}
