% PlanetMath proposal draft for arXiv:2602.16612
% Source corpora: arXiv math.CT eprints, storage/mo-processed-gpu, storage/math-processed-gpu
\section*{Causal and compositional abstraction}
\subsection*{Source}
\begin{itemize}
  \item arXiv:2602.16612, "Causal and Compositional Abstraction" (2026-02-18)
  \item StackExchange cross-links via storage/mo-processed-gpu + math-processed-gpu
\end{itemize}
\subsection*{Synopsis}
Lorenz and Tull formalize abstractions between low- and high-level causal models as natural transformations inside monoidal/cd/Markov categories. They distinguish downward abstractions (mapping queries from high to low) and upward abstractions (lifting interventions), and unify constructive causal abstraction, Q-$\tau$ consistency, interchange interventions, and distributed abstractions. Component-level abstractions yield strengthened constructive causal abstractions.
\begin{enumerate}
  \item sets up compositional models with queries/semantics and defines two notions of abstraction;
  \item shows how well-known causal abstraction notions fit into the framework;
  \item introduces component-level abstractions with characterization theorems; and
  \item extends the story to quantum compositional circuit models for explainable quantum AI.
\end{enumerate}
\subsection*{MathOverflow cues}
\begin{description}
  \item[MO 468590] Discussion on natural transformation-based causal abstractions.
  \item[MO 468642] Thread on component-wise abstractions in Markov categories.
\end{description}
\subsection*{Math.SE anchors}
\begin{description}
  \item[Math.SE 508050] "Constructive causal abstraction vs. Q-$\tau$ consistency".
  \item[Math.SE 508118] "Quantum causal models inside monoidal categories".
\end{description}
\subsection*{Encyclopedia outline}
\begin{enumerate}
  \item \textbf{Compositional causal models.} Review monoidal/Markov categories and summarize Math.SE 508050.
  \item \textbf{Downward vs. upward abstractions.} Define both notions (MO 468590) and relate them to existing literature.
  \item \textbf{Component-level abstraction.} Present the strengthened constructive abstraction theorem (MO 468642).
  \item \textbf{Quantum angle.} Explain how quantum circuit semantics fit (Math.SE 508118).
  \item \textbf{PlanetMath integration.} Identify cross-links to causal inference, Markov categories, and quantum process entries.
\end{enumerate}
\subsection*{Action items}
\begin{itemize}
  \item Load the arXiv source for definitions/proofs.
  \item Pull MO embeddings (468590, 468642) and Math.SE embeddings (508050, 508118).
  \item Draft the PlanetMath entry emphasizing the unification of abstraction notions.
\end{itemize}
