% PlanetMath proposal draft for arXiv:2602.11943
% Source corpora: arXiv math.CT eprints, storage/mo-processed-gpu, storage/math-processed-gpu
\section*{Cylinder simplicial DG rings and Kan conditions}
\subsection*{Source}
\begin{itemize}
  \item arXiv:2602.11943, ``The Cylinder Simplicial DG Ring'' (February 2026)
  \item MathOverflow threads on DG Morita equivalence, semi-free cdga homology, and path objects in CDGA (IDs 245448, 111688, 239240)
  \item Math.StackExchange questions on bar constructions and Kähler differentials for DG algebras (IDs 4288818, 4216165)
\end{itemize}
\subsection*{Synopsis}
Yekutieli defines the $q$-th cylinder DG ring $\Cyl_q(B)$ for any DG ring $B$, showing that $\Cyl(B)$ forms a simplicial DG ring encoding homotopies of DG ring maps. If $A$ is semi-free, then $\Hom(A,\Cyl(B))$ is a Kan complex. The proof constructs DG representatives for simplicial horns, verifies lifting conditions, and argues that these constructions extend to DG categories.
\begin{enumerate}
  \item For each $q\ge 0$, builds $\Cyl_q(B)$ so that $\Cyl_1(B)$ recovers Keller's cylinder/path object and higher $q$ assemble into a simplicial DG ring.
  \item Shows that $\Hom(A,\Cyl(B))$ inherits a simplicial structure whose $n$-simplices are DG maps $A\to\Cyl_n(B)$.
  \item Proves the Kan lifting property when $A$ is semi-free by representing horns via DG rings $\mathrm{N}(\Lambda^q_i,B)$ fit into exact sequences.
  \item Interprets the horn fillings explicitly in terms of DG algebra operations, which gives constructive homotopies of DG ring maps.
  \item Suggests extensions to DG categories and derived completion contexts.
\end{enumerate}
\subsection*{MathOverflow cues}
\begin{description}
  \item[MO 245448] Connects derived equivalence with DG Morita equivalence, underscoring why cylinder/path objects are important for DG homotopy theory.
  \item[MO 111688] Discusses cyclic homology of semi-free cdgas, motivating the semi-free hypothesis in the Kan condition.
  \item[MO 239240] Provides examples of small homotopy limits in CDGAs using path objects, paralleling the cylinder construction here.
\end{description}
\subsection*{Math.SE anchors}
\begin{description}
  \item[Math.SE 4288818] Asks why the bar construction of a DG algebra is a coalgebra—useful background for interpreting the simplicial structure on $\Cyl(B)$.
  \item[Math.SE 4216165] Seeks definitions of Kähler differentials for DG algebras, aligning with the technical tools used in the horn-representing DG rings.
\end{description}
\subsection*{Encyclopedia outline}
\begin{enumerate}
  \item Definition of the $q$-th cylinder DG ring and relation to Keller's cylinder.
  \item Simplicial structure on $\Cyl(B)$ and induced simplicial sets $\Hom(A,\Cyl(B))$.
  \item Construction of horn DG rings $\mathrm{N}(\Lambda^q_i,B)$ and the Kan lifting proof for semi-free sources.
  \item Examples and remarks on extending to DG categories or derived algebraic geometry.
\end{enumerate}
\subsection*{Action items}
\begin{itemize}
  \item Write a PlanetMath entry describing $\Cyl(B)$, including explicit formulas for $q$-simplices and degeneracy faces.
  \item Explain the horn-representing DG rings and why semi-free $A$ guarantees liftings.
  \item Cross-link to entries on DG homotopy, model structures on DG rings, and Keller's path object.
  \item Mention potential DG-category generalizations and references for further reading.
\end{itemize}
