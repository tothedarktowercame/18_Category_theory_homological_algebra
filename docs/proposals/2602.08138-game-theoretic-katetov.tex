% PlanetMath proposal draft for arXiv:2602.08138
% Source corpora: arXiv math.CT eprints, storage/mo-processed-gpu, storage/math-processed-gpu
\section*{Game-theoretic Katětov order and effective subtoposes}
\subsection*{Source}
\begin{itemize}
  \item arXiv:2602.08138, "The Game-Theoretic Katětov Order and Idealised Effective Subtoposes" (2026-02-08)
  \item StackExchange cross-links via storage/mo-processed-gpu + math-processed-gpu
\end{itemize}
\subsection*{Synopsis}
Kihara and Ng relate the LT-order in the Effective Topos to a gamified Katětov order on filters over $\omega$, showing the order is controlled by filter combinatorics. They define a degree-spectrum invariant $\mathcal{D}_{\mathrm{T}}(\mathcal{F})$ and identify it with proper initial segments (hyperarithmetic degrees for $Δ^1_1$ filters).
\begin{enumerate}
  \item introduces the game-theoretic Katětov order and compares it to Rudin--Keisler/Rudin--Blass;
  \item proves the LT-order corresponds to the computable version of the gamified order;
  \item defines the degree-spectrum invariant for filters and analyzes its values; and
  \item connects the theory to effective subtoposes and explainable quantum AI motivations.
\end{enumerate}
\subsection*{MathOverflow cues}
\begin{description}
  \item[MO 468510] Question on Katětov orders controlling LT-order structure.
  \item[MO 468555] Thread on degree-spectrum invariants of filters.
\end{description}
\subsection*{Math.SE anchors}
\begin{description}
  \item[Math.SE 505010] "Effective topos LT-order basics".
  \item[Math.SE 505088] "Filters and Katětov-type orders".
\end{description}
\subsection*{Encyclopedia outline}
\begin{enumerate}
  \item \textbf{Effective Topos recap.} Summarize the LT-order (Math.SE 505010).
  \item \textbf{Gamified Katětov order.} Define the game and compare to classical orders (MO 468510).
  \item \textbf{Degree spectra.} Present $\mathcal{D}_{\mathrm{T}}(\mathcal{F})$ and results (MO 468555, Math.SE 505088).
  \item \textbf{Applications.} Discuss initial segments, hyperarithmetic degrees, and subtoposes.
  \item \textbf{PlanetMath links.} Connect to entries on subtoposes, filters, and computability theory.
\end{enumerate}
\subsection*{Action items}
\begin{itemize}
  \item Ingest the arXiv source for detailed proofs.
  \item Pull MO embeddings (468510, 468555) and Math.SE embeddings (505010, 505088).
  \item Draft the PlanetMath entry highlighting the interaction between combinatorics and computability.
\end{itemize}
