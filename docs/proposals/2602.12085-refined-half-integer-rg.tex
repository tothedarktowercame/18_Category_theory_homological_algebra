% PlanetMath proposal draft for arXiv:2602.12085
% Source corpora: arXiv math.CT eprints, storage/mo-processed-gpu, storage/math-processed-gpu
\section*{Refined half-integer condition on RG flows}
\subsection*{Source}
\begin{itemize}
  \item arXiv:2602.12085, "Refined half-integer condition on RG flows" (2026-02-12)
  \item StackExchange cross-links via storage/mo-processed-gpu + math-processed-gpu
\end{itemize}
\subsection*{Synopsis}
Ken Kikuchi refines the half-integer constraint controlling conformal dimensions of symmetry objects that persist across renormalization group flows with defects. Using braided symmetry categories, he derives necessary conditions under which the ultraviolet and infrared conformal dimensions sum to a half-integer and solves example flows showcasing the constraint.
\begin{enumerate}
  \item Reinterprets the half-integer condition via monoidal/ braided structures on symmetry categories.
  \item Derives necessary conditions for half-integer sums of conformal dimensions.
  \item Applies the refined constraint to specific RG flows with defects.
\end{enumerate}
\subsection*{MathOverflow cues}
\begin{description}
  \item[MO 479811] Symmetry categories and RG defects enforcing half-integer sums.
  \item[MO 479845] Examples of braided symmetry categories along RG flows.
\end{description}
\subsection*{Math.SE anchors}
\begin{description}
  \item[Math.SE 509315] "Computing UV/IR dimensions in braided symmetry categories".
  \item[Math.SE 509347] "Half-integer constraint checks" -- explicit flows.
\end{description}
\subsection*{Encyclopedia outline}
\begin{enumerate}
  \item \textbf{Background.} Summarise symmetry categories, RG defects, and the previous half-integer rule (MO 479811).
  \item \textbf{Refined condition.} Present Kikuchi's necessary condition (Math.SE 509315).
  \item \textbf{Examples.} Walk through solved flows verifying/refuting the constraint (MO 479845, Math.SE 509347).
  \item \textbf{Implications.} Discuss how additional categorical structures constrain RG flows.
  \item \textbf{Outlook.} Note possible applications to categorical anomalies.
\end{enumerate}
\subsection*{Action items}
\begin{itemize}
  \item Import the arXiv source for precise statements and solved examples.
  \item Link MO 479811/479845 and Math.SE 509315/509347 discussions.
  \item Emphasise the categorical viewpoint on RG constraints in PlanetMath.
\end{itemize}
