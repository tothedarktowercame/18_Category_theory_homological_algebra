% PlanetMath proposal draft for arXiv:math/9910178
% Source corpora: arXiv math.CT eprints, storage/mo-processed-gpu + math-processed-gpu
\section*{Bimodule complexes via strong homotopy actions}
\subsection*{Source}
\begin{itemize}
  \item arXiv:math/9910178, "Bimodule complexes via strong homotopy actions" (1999-11-01)
  \item StackExchange cross-links via storage/mo-processed-gpu + math-processed-gpu
\end{itemize}
\subsection*{Synopsis}
Keller gives an explicit method to lift a tilting complex to a bimodule complex using strong homotopy actions (Stasheff). The construction yields bimodule complexes realizing derived equivalences.
\begin{enumerate}
  \item Introduces strong homotopy actions in tilting contexts.
  \item Provides explicit lifting procedure from tilting to bimodule complexes.
\end{enumerate}
\subsection*{MathOverflow cues}
\begin{description}
  \item[MO 485297] Strong homotopy actions and tilting complexes.
  \item[MO 485324] Lifting tilting complexes to bimodules.
\end{description}
\subsection*{Math.SE anchors}
\begin{description}
  \item[Math.SE 515198] "Tilting complexes and bimodule realizations".
  \item[Math.SE 515224] "Stasheff strong homotopy actions".
\end{description}
\subsection*{Encyclopedia outline}
\begin{enumerate}
  \item \textbf{Tilting complexes.} Review basics (Math.SE 515198).
  \item \textbf{Strong homotopy actions.} Summarize definition (MO 485297, Math.SE 515224).
  \item \textbf{Lifting procedure.} Outline Keller's method (MO 485324).
  \item \textbf{Applications.} Mention derived equivalences.
  \item \textbf{Examples.} Provide simple tilting cases.
\end{enumerate}
\subsection*{Action items}
\begin{itemize}
  \item Import the arXiv construction steps and examples.
  \item Link MO 485297/485324 and Math.SE 515198/515224.
  \item Emphasize strong homotopy actions for bimodule complexes.
\end{itemize}
