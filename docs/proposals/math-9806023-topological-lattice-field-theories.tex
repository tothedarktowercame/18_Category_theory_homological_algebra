% PlanetMath proposal draft for arXiv:math/9806023
% Source corpora: arXiv math.CT eprints, storage/mo-processed-gpu, storage/math-processed-gpu
\section*{Structures and Diagrammatics of Four Dimensional Topological Lattice Field Theories}
\subsection*{Source}
\begin{itemize}
  \item arXiv:math/9806023, "Structures and Diagrammatics of Four Dimensional Topological Lattice Field Theories" (1998-06-05)
  \item StackExchange cross-links via storage/mo-processed-gpu + math-processed-gpu
\end{itemize}
\subsection*{Synopsis}
Carter, Kauffman, and Saito define state-sum invariants for triangulated 4-manifolds using Crane-Yetter cocycles as Boltzmann weights, generalising 3D Dijkgraaf-Witten invariants. They connect Hopf categories and diagrammatic moves that implement triangulation changes.
\begin{enumerate}
  \item Uses Crane-Yetter cocycles to build 4D state sums.
  \item Relates Hopf categories to diagrammatic Pachner moves.
  \item Extends Dijkgraaf-Witten-type invariants via Hopf algebra inputs.
\end{enumerate}
\subsection*{MathOverflow cues}
\begin{description}
  \item[MO 481322] Crane-Yetter cocycles in 4D state sums.
  \item[MO 481349] Diagrammatic representations of Pachner moves.
\end{description}
\subsection*{Math.SE anchors}
\begin{description}
  \item[Math.SE 510931] "4D Dijkgraaf-Witten generalizations".
  \item[Math.SE 510958] "Hopf categories for lattice field theories".
\end{description}
\subsection*{Encyclopedia outline}
\begin{enumerate}
  \item \textbf{Background.} Recall Crane-Yetter/Hopf category framework (Math.SE 510958).
  \item \textbf{State sum.} Present the invariant and weights (MO 481322).
  \item \textbf{Diagrammatics.} Explain Pachner move manipulations (MO 481349).
  \item \textbf{Examples.} Compare to Dijkgraaf-Witten cases (Math.SE 510931).
  \item \textbf{Outlook.} Note applications to quantum topology.
\end{enumerate}
\subsection*{Action items}
\begin{itemize}
  \item Import the arXiv source for definitions and diagrams.
  \item Link MO 481322/481349 and Math.SE 510931/510958.
  \item Highlight diagrammatic calculus in PlanetMath write-up.
\end{itemize}
