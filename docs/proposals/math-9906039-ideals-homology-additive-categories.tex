% PlanetMath proposal draft for arXiv:math/9906039
% Source corpora: arXiv math.CT eprints, storage/mo-processed-gpu, storage/math-processed-gpu
\section*{On ideals and homology in additive categories}
\subsection*{Source}
\begin{itemize}
  \item arXiv:math/9906039, "On Ideals and Homology in Additive Categories" (1999-06-07)
  \item StackExchange cross-links via storage/mo-processed-gpu + math-processed-gpu
\end{itemize}
\subsection*{Synopsis}
Ionescu uses ideals to define homological functors on additive categories. In abelian settings the ideals tied to universal objects are principal, recovering homology groups; applications include derived categories and functors.
\begin{enumerate}
  \item Defines homological functors via ideals in additive categories.
  \item Shows reduction to standard homology in abelian categories.
  \item Applies to derived categories and functors.
\end{enumerate}
\subsection*{MathOverflow cues}
\begin{description}
  \item[MO 484405] Ideals and homological functors in additive categories.
  \item[MO 484432] Principal ideals and universal objects.
\end{description}
\subsection*{Math.SE anchors}
\begin{description}
  \item[Math.SE 514245] "Ideals defining homology".
  \item[Math.SE 514272] "Derived categories via ideals".
\end{description}
\subsection*{Encyclopedia outline}
\begin{enumerate}
  \item \textbf{Additive categories.} Recall ideals and universal objects (Math.SE 514245).
  \item \textbf{Homological functors.} Describe construction (MO 484405).
  \item \textbf{Abelian reduction.} Explain principal ideal case (MO 484432).
  \item \textbf{Applications.} Mention derived categories/functors (Math.SE 514272).
  \item \textbf{Examples.} Provide simple additive categories.
\end{enumerate}
\subsection*{Action items}
\begin{itemize}
  \item Import the arXiv text for definitions and applications.
  \item Link MO 484405/484432 and Math.SE 514245/514272.
  \item Emphasize ideal-based homology interpretation.
\end{itemize}
