% PlanetMath proposal draft for arXiv:math/9911242
% Source corpora: arXiv math.CT eprints, storage/mo-processed-gpu, storage/math-processed-gpu
\section*{Noetherian hereditary categories satisfying Serre duality}
\subsection*{Source}
\begin{itemize}
  \item arXiv:math/9911242, "Noetherian hereditary categories satisfying Serre duality" (1999-11-30)
  \item StackExchange cross-links via storage/mo-processed-gpu + math-processed-gpu
\end{itemize}
\subsection*{Synopsis}
Reiten and Van den Bergh classify noetherian hereditary abelian categories satisfying Serre duality (Bondal-Kapranov sense) and thus saturated noetherian hereditary categories. Without projectives/injectives, Serre duality is equivalent to the existence of almost split sequences.
\begin{enumerate}
  \item Classifies noetherian hereditary abelian categories with Serre duality.
  \item Shows Serre duality $
leftrightarrow$ almost split sequences when no proj/inj.
\end{enumerate}
\subsection*{MathOverflow cues}
\begin{description}
  \item[MO 484949] Classification of hereditary categories with Serre duality.
  \item[MO 484976] Serre duality and almost split sequences.
\end{description}
\subsection*{Math.SE anchors}
\begin{description}
  \item[Math.SE 514735] "Examples of Serre-dual hereditary categories".
  \item[Math.SE 514761] "Almost split sequences vs. Serre duality".
\end{description}
\subsection*{Encyclopedia outline}
\begin{enumerate}
  \item \textbf{Definitions.} Review noetherian hereditary categories and Serre duality (Math.SE 514735).
  \item \textbf{Classification.} Summarize main theorem (MO 484949).
  \item \textbf{Almost split sequences.} Present equivalence when no proj/inj (MO 484976, Math.SE 514761).
  \item \textbf{Saturated categories.} Connect to saturated hereditary cases.
  \item \textbf{Examples.} Provide typical derived categories.
\end{enumerate}
\subsection*{Action items}
\begin{itemize}
  \item Import arXiv classification results and equivalence statements.
  \item Link MO 484949/484976 and Math.SE 514735/514761.
  \item Highlight Serre duality implications in PlanetMath entry.
\end{itemize}
