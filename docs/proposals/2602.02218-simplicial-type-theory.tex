% PlanetMath proposal draft for arXiv:2602.02218
% Source corpora: arXiv math.CT eprints, storage/mo-processed-gpu, storage/math-processed-gpu
\section*{$\infty$-categories inside simplicial type theory}
\subsection*{Source}
\begin{itemize}
  \item arXiv:2602.02218, "The $\infty$-category of $\infty$-categories in simplicial type theory" (2026-02-02)
  \item StackExchange cross-links via storage/mo-processed-gpu + math-processed-gpu
\end{itemize}
\subsection*{Synopsis}
Gratzer, Weinberger, and Buchholtz use simplicial type theory (STT) to construct the $\infty$-category of $\infty$-categories, giving a type-theoretic proof of straightening/unstraightening. They leverage techniques from cubical type theory to build the $\infty$-category of spaces, then extend them to directed structure identity principles.
\begin{enumerate}
  \item reviews STT foundations and recent cubical techniques for constructing spaces;
  \item builds the $\infty$-category of $\infty$-categories entirely within STT;
  \item recovers straightening--unstraightening and the structure identity principle in this setting; and
  \item presents new directed structure homomorphism examples.
\end{enumerate}
\subsection*{MathOverflow cues}
\begin{description}
  \item[MO 467355] Question on implementing unstraightening in simplicial type theory.
  \item[MO 467402] Discussion of directed structure identity principles in type-theoretic $\infty$-categories.
\end{description}
\subsection*{Math.SE anchors}
\begin{description}
  \item[Math.SE 502880] "From cubical to simplicial type theory for $\infty$-categories".
  \item[Math.SE 503022] "Straightening/unstraightening proofs internal to STT".
\end{description}
\subsection*{Encyclopedia outline}
\begin{enumerate}
  \item \textbf{STT background.} Summarize the syntax/semantics and cite Math.SE 502880.
  \item \textbf{Spaces in STT.} Recount the cubical-inspired construction used as a stepping stone.
  \item \textbf{$\infty$Cat in STT.} Detail the main construction and reference MO 467355.
  \item \textbf{Structure identity principles.} Discuss directed versions and MO 467402/Math.SE 503022.
  \item \textbf{Applications.} Note new internal examples and how they interact with PlanetMath's type-theory pages.
\end{enumerate}
\subsection*{Action items}
\begin{itemize}
  \item Import the arXiv source and extract the type-theoretic rules.
  \item Pull MO embeddings (467355, 467402) and Math.SE embeddings (502880, 503022).
  \item Draft the PlanetMath entry outlining comparisons with cubical HoTT.
\end{itemize}
