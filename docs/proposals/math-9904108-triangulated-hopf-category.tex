% PlanetMath proposal draft for arXiv:math/9904108
% Source corpora: arXiv math.CT eprints, storage/mo-processed-gpu, storage/math-processed-gpu
\section*{The triangulated Hopf category $n_{+}SL(2)$}
\subsection*{Source}
\begin{itemize}
  \item arXiv:math/9904108, "The triangulated Hopf category $n_{+}SL(2)$" (1999-04-21)
  \item StackExchange cross-links via storage/mo-processed-gpu + math-processed-gpu
\end{itemize}
\subsection*{Synopsis}
Lyubashenko studies an equivariant derived category equipped with multiplication/comultiplication functors forming a triangulated Hopf category related to $SL(2)$. He proves coherence equations for structure isomorphisms, showing the Hopf category is monoidal.
\begin{enumerate}
  \item Constructs an $SL(2)$-equivariant triangulated Hopf category.
  \item Provides multiplication/comultiplication functors with structure isomorphisms.
  \item Establishes coherence and monoidality.
\end{enumerate}
\subsection*{MathOverflow cues}
\begin{description}
  \item[MO 484332] Hopf categories from derived categories.
  \item[MO 484359] Coherence in triangulated Hopf categories.
\end{description}
\subsection*{Math.SE anchors}
\begin{description}
  \item[Math.SE 514172] "Examples of triangulated Hopf categories".
  \item[Math.SE 514198] "Equivariant derived categories with Hopf structures".
\end{description}
\subsection*{Encyclopedia outline}
\begin{enumerate}
  \item \textbf{Hopf categories.} Review definitions (Math.SE 514172).
  \item \textbf{Construction.} Outline $n_{+}SL(2)$ example (MO 484332).
  \item \textbf{Coherence.} Summarize structure isomorphisms (MO 484359, Math.SE 514198).
  \item \textbf{Monoidality.} Explain resulting monoidal structure.
  \item \textbf{Applications.} Mention links to representation theory.
\end{enumerate}
\subsection*{Action items}
\begin{itemize}
  \item Import arXiv content for functorial structure and coherence equations.
  \item Link MO 484332/484359 and Math.SE 514172/514198.
  \item Emphasize triangulated Hopf category structure.
\end{itemize}
