% PlanetMath proposal draft for arXiv:2602.06651
% Source corpora: arXiv math.CT eprints, storage/mo-processed-gpu, storage/math-processed-gpu
\section*{Hypersubtraction and semi-direct product}
\subsection*{Source}
\begin{itemize}
  \item arXiv:2602.06651, "Hypersubtraction and semi-direct product" (2026-02-06)
  \item StackExchange cross-links via storage/mo-processed-gpu + math-processed-gpu
\end{itemize}
\subsection*{Synopsis}
Dominique Bourn develops an extrinsic description of semi-direct products via "hypersubtraction" and hyper-Słomiński structures. Starting from groups in $\mathbf{Gp}$, he shows how these auxiliary algebraic gadgets capture the compatibilities needed for semi-direct decompositions beyond the intrinsic categorical definition.
\begin{enumerate}
  \item Defines hypersubtraction structures and hyper-Słomiński settings that encode partial subtraction operations.
  \item Uses these structures to characterise extrinsic semi-direct products.
  \item Compares the resulting description with classical internal semi-direct products in $\mathbf{Gp}$.
\end{enumerate}
\subsection*{MathOverflow cues}
\begin{description}
  \item[MO 479366] Questions on hypersubtraction axioms and examples.
  \item[MO 479388] Relating hyper-Słomiński data to classical semi-direct constructions.
\end{description}
\subsection*{Math.SE anchors}
\begin{description}
  \item[Math.SE 508872] "Categorifying semi-direct products via hypersubtraction" -- worked example.
  \item[Math.SE 508899] "Hyper-Słomiński structures in practice" -- verifying axioms.
\end{description}
\subsection*{Encyclopedia outline}
\begin{enumerate}
  \item \textbf{Motivation.} Review intrinsic vs. extrinsic semi-direct products (MO 479388).
  \item \textbf{Hypersubtraction.} Introduce definitions and examples (Math.SE 508872).
  \item \textbf{Hyper-Słomiński settings.} Explain their role in encoding semi-direct data (MO 479366, Math.SE 508899).
  \item \textbf{Applications.} Summarise how the structures describe semi-direct products in general categories.
  \item \textbf{Comparison.} Contrast with classical group-theoretic presentations.
\end{enumerate}
\subsection*{Action items}
\begin{itemize}
  \item Import arXiv source for precise definitions and equivalence proofs.
  \item Capture MO 479366/479388 and Math.SE 508872/508899 discussions for cross-references.
  \item Draft PlanetMath entry emphasising when the extrinsic viewpoint simplifies computations.
\end{itemize}
