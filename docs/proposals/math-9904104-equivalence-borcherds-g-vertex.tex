% PlanetMath proposal draft for arXiv:math/9904104
% Source corpora: arXiv math.CT eprints, storage/mo-processed-gpu, storage/math-processed-gpu
\section*{Equivalence of Borcherds $G$-vertex algebras and axiomatic vertex algebras}
\subsection*{Source}
\begin{itemize}
  \item arXiv:math/9904104, "Equivalence of Borcherds G-Vertex Algebras and Axiomatic Vertex Algebras" (1999-04-20)
  \item StackExchange cross-links via storage/mo-processed-gpu + math-processed-gpu
\end{itemize}
\subsection*{Synopsis}
Craig Snydal constructs vertex algebras from Borcherds' vertex groups via singular multilinear maps parameterized by trees, obtaining natural notions of rationality, commutativity, and associativity, and proving equivalence between Borcherds $G$-vertex algebras and axiomatic vertex algebras in a relaxed multilinear category.
\begin{enumerate}
  \item Builds singular multilinear maps from vertex groups.
  \item Derives generalized rationality/commutativity/associativity.
  \item Proves categorical equivalence of two vertex algebra notions.
\end{enumerate}
\subsection*{MathOverflow cues}
\begin{description}
  \item[MO 484096] Vertex groups and singular maps.
  \item[MO 484123] Equivalence of $G$-vertex and axiomatic vertex algebras.
\end{description}
\subsection*{Math.SE anchors}
\begin{description}
  \item[Math.SE 513870] "Tree-parameterized singular maps".
  \item[Math.SE 513896] "Comparing Borcherds and axiomatic vertex algebras".
\end{description}
\subsection*{Encyclopedia outline}
\begin{enumerate}
  \item \textbf{Vertex groups.} Present Borcherds' setup (Math.SE 513870).
  \item \textbf{Singular maps.} Describe tree-parameterized construction (MO 484096).
  \item \textbf{Equivalence theorem.} Explain categorical equivalence (MO 484123, Math.SE 513896).
  \item \textbf{Consequences.} Discuss rationality/commutativity as natural corollaries.
  \item \textbf{Examples.} Provide sample vertex groups.
\end{enumerate}
\subsection*{Action items}
\begin{itemize}
  \item Import the arXiv paper for the singular map construction and equivalence proof.
  \item Link MO 484096/484123 and Math.SE 513870/513896.
  \item Emphasize categorical equivalence for PlanetMath entry.
\end{itemize}
