% PlanetMath proposal draft for arXiv:math/9909030
% Source corpora: arXiv math.CT eprints, storage/mo-processed-gpu, storage/math-processed-gpu
\section*{Grothendieck categories}
\subsection*{Source}
\begin{itemize}
  \item arXiv:math/9909030, "Grothendieck Categories" (1999-09-05)
  \item StackExchange cross-links via storage/mo-processed-gpu + math-processed-gpu
\end{itemize}
\subsection*{Synopsis}
Garkusha systematizes methods and results in the theory of Grothendieck categories, explaining how to apply them to rings and modules.
\begin{enumerate}
  \item Surveys core tools of Grothendieck category theory.
  \item Illustrates applications to ring and module theory.
\end{enumerate}
\subsection*{MathOverflow cues}
\begin{description}
  \item[MO 484805] Key theorems in Grothendieck categories.
  \item[MO 484832] Applications to module categories.
\end{description}
\subsection*{Math.SE anchors}
\begin{description}
  \item[Math.SE 514608] "Grothendieck category techniques".
  \item[Math.SE 514634] "Using Grothendieck categories for rings".
\end{description}
\subsection*{Encyclopedia outline}
\begin{enumerate}
  \item \textbf{Definitions.} Reintroduce AB5, generators, etc. (Math.SE 514608).
  \item \textbf{Methods.} Summarize principal results (MO 484805).
  \item \textbf{Applications.} Discuss rings/modules (MO 484832, Math.SE 514634).
  \item \textbf{Examples.} Provide canonical Grothendieck categories.
  \item \textbf{References.} Note systematic approach.
\end{enumerate}
\subsection*{Action items}
\begin{itemize}
  \item Import arXiv survey notes for reference.
  \item Link MO 484805/484832 and Math.SE 514608/514634.
  \item Highlight application pathways.
\end{itemize}
