% PlanetMath proposal draft for arXiv:2602.19480
% Source corpora: arXiv math.CT eprints, storage/mo-processed-gpu, storage/math-processed-gpu
\section*{Condensed limits via span logoi}
\subsection*{Source}
\begin{itemize}
  \item arXiv:2602.19480, "Condensed limits via span logoi" (2026-02-20)
  \item StackExchange cross-links via storage/mo-processed-gpu + math-processed-gpu
\end{itemize}
\subsection*{Synopsis}
The manuscript computes limits/colimits in condensed mathematics using span logoi to encode condensed symmetry. Key steps:
\begin{enumerate}
  \item construct span logoi from condensed groups and record how limits pull back along spans;
  \item prove a condensed Beck--Chevalley theorem ensuring the span calculus respects base-change;
  \item treat condensed tori and condensed fields as working examples; and
  \item connect the story to Galois symmetries highlighted in MO 466647 + MO 467599.
\end{enumerate}
\subsection*{MathOverflow cues}
\begin{description}
  \item[MO 466647] Picard/Picard span discussion relevant to condensed tori.
  \item[MO 467599] Finite-group spans controlling condensed symmetry constraints.
\end{description}
\subsection*{Math.SE anchors}
\begin{description}
  \item[Math.SE 498240] "Condensed sets vs. logoi".
  \item[Math.SE 499776] "Span correspondences in condensed mathematics".
\end{description}
\subsection*{Encyclopedia outline}
\begin{enumerate}
  \item Review condensed logoi and summarize Math.SE 498240.
  \item Define span logoi and state the condensed Beck--Chevalley theorem.
  \item Carry out the condensed torus example, explaining how MO 466647 enters.
  \item Describe the Galois symmetry application (MO 467599).
  \item Note how the explicit limit formulas feed back into condensed field computations.
\end{enumerate}
\subsection*{Action items}
Sync arXiv source, fetch MO embeddings (466647, 467599), fetch Math.SE embeddings (498240, 499776), and expand the PlanetMath entry per outline.
