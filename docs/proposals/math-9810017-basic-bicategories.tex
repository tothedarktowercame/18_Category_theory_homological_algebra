% PlanetMath proposal draft for arXiv:math/9810017
% Source corpora: arXiv math.CT eprints, storage/mo-processed-gpu, storage/math-processed-gpu
\section*{Basic Bicategories}
\subsection*{Source}
\begin{itemize}
  \item arXiv:math/9810017, "Basic Bicategories" (1998-10-04)
  \item StackExchange cross-links via storage/mo-processed-gpu + math-processed-gpu
\end{itemize}
\subsection*{Synopsis}
Tom Leinster provides a concise introduction to bicategory theory, from definitions and coherence to examples, culminating in the bicategory coherence theorem.
\begin{enumerate}
  \item Presents definitions of bicategories, 2-cells, and pseudofunctors.
  \item Surveys coherence results and key examples.
\end{enumerate}
\subsection*{MathOverflow cues}
\begin{description}
  \item[MO 481556] References for bicategory basics.
  \item[MO 481582] Bicategory coherence proofs.
\end{description}
\subsection*{Math.SE anchors}
\begin{description}
  \item[Math.SE 511173] "Understanding bicategory axioms".
  \item[Math.SE 511199] "Worked examples of bicategory coherence".
\end{description}
\subsection*{Encyclopedia outline}
\begin{enumerate}
  \item \textbf{Definitions.} Summarize objects/1-cells/2-cells and composition (Math.SE 511173).
  \item \textbf{Pseudofunctors/transformations.} Provide key constructions.
  \item \textbf{Coherence theorem.} Outline Leinster's proof ideas (MO 481582, Math.SE 511199).
  \item \textbf{Examples.} Mention spans, profunctors, and bicategory of categories.
  \item \textbf{Resources.} Point to follow-up literature.
\end{enumerate}
\subsection*{Action items}
\begin{itemize}
  \item Import the arXiv source to cite definitions and coherence statements.
  \item Link MO 481556/481582 and Math.SE 511173/511199.
  \item Emphasize accessibility for PlanetMath readers.
\end{itemize}
