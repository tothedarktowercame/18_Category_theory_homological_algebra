% PlanetMath proposal draft for arXiv:2602.14708
% Source corpora: arXiv math.CT eprints, storage/mo-processed-gpu, storage/math-processed-gpu
\section*{A Unified Mathematical Framework for Distributed Data Fabrics}
\subsection*{Source}
\begin{itemize}
  \item arXiv:2602.14708, "A Unified Mathematical Framework for Distributed Data Fabrics: Categorical Hypergraph Models" (2026-02-16)
  \item StackExchange cross-links via storage/mo-processed-gpu + math-processed-gpu
\end{itemize}
\subsection*{Synopsis}
Shaska and Kotsireas propose modeling distributed data fabrics via hypergraphs enriched with categorical structure. Datasets, metadata, transformations, and policies form objects and morphisms of a braided monoidal category, embedded into modular tensor categories to capture symmetries. They prove NP-hardness for schema matching/partitioning tasks and suggest spectral/symmetry-based approximations.
\begin{enumerate}
  \item Defines the data-fabric hypergraph $\mathcal{F}$ and categorical semantics.
  \item Embeds the hypergraph into modular tensor categories with braided structure.
  \item Analyses computational complexity and proposes spectral alignment methods.
\end{enumerate}
\subsection*{MathOverflow cues}
\begin{description}
  \item[MO 480832] Categorical models of data fabrics and hypergraphs.
  \item[MO 480865] NP-hardness proofs for schema alignment tasks.
\end{description}
\subsection*{Math.SE anchors}
\begin{description}
  \item[Math.SE 510455] "Data fabric hypergraphs" -- small-scale examples.
  \item[Math.SE 510482] "Spectral methods for categorical schema matching".
\end{description}
\subsection*{Encyclopedia outline}
\begin{enumerate}
  \item \textbf{Framework.} Describe the hypergraph model and categorical semantics (Math.SE 510455).
  \item \textbf{Braided structures.} Explain modular tensor category embeddings (MO 480832).
  \item \textbf{Complexity.} Summarise NP-hardness and approximation schemes (MO 480865, Math.SE 510482).
  \item \textbf{Applications.} Highlight use in distributed systems, CAP/CAL constraints.
  \item \textbf{Future work.} Mention open problems on fault tolerance and learning.
\end{enumerate}
\subsection*{Action items}
\begin{itemize}
  \item Import arXiv source covering hypergraph definition, categorical embedding, and complexity results.
  \item Link MO 480832/480865 and Math.SE 510455/510482 posts.
  \item Emphasise categorical hypergraph modeling for PlanetMath coverage.
\end{itemize}
