% PlanetMath proposal draft for arXiv:math/9906032
% Source corpora: arXiv math.CT eprints, storage/mo-processed-gpu, storage/math-processed-gpu
\section*{Berikashvili's functor $D$ and the deformation equation}
\subsection*{Source}
\begin{itemize}
  \item arXiv:math/9906032, "Berikashvili's functor D and the deformation equation" (1999-06-05)
  \item StackExchange cross-links via storage/mo-processed-gpu + math-processed-gpu
\end{itemize}
\subsection*{Synopsis}
Huebschmann relates Berikashvili's functor $D$ (defined via twisting cochains) to deformation theory, gauge theory, Chen's formal power series connections, and the master equation. Twisting cochains and classifying twisting cochains unify these topics.
\begin{enumerate}
  \item Reviews Berikashvili's functor $D$ and twisting cochains.
  \item Connects $D$ to deformation theory and gauge/Maurer-Cartan equations.
  \item Explains classification via classifying twisting cochains.
\end{enumerate}
\subsection*{MathOverflow cues}
\begin{description}
  \item[MO 484658] Twisting cochains and Berikashvili's functor.
  \item[MO 484685] Master equation interpretations.
\end{description}
\subsection*{Math.SE anchors}
\begin{description}
  \item[Math.SE 514455] "Understanding functor $D$".
  \item[Math.SE 514481] "Twisting cochains in deformation theory".
\end{description}
\subsection*{Encyclopedia outline}
\begin{enumerate}
  \item \textbf{Twisting cochains.} Introduce basics (Math.SE 514455).
  \item \textbf{Functor $D$.} Describe construction and properties (MO 484658).
  \item \textbf{Deformation/master equation.} Explain connections (MO 484685, Math.SE 514481).
  \item \textbf{Applications.} Mention gauge theory and Chen connections.
  \item \textbf{Classifying twisting cochains.} Summarize unifying viewpoint.
\end{enumerate}
\subsection*{Action items}
\begin{itemize}
  \item Import arXiv sections on functor $D$ and twisting cochains.
  \item Link MO 484658/484685 and Math.SE 514455/514481.
  \item Emphasize the unifying role of twisting cochains.
\end{itemize}
