% PlanetMath proposal draft for arXiv:2602.01292
% Source corpora: arXiv math.CT eprints, storage/mo-processed-gpu, storage/math-processed-gpu
\section*{Factorization algebras in isolability settings}
\subsection*{Source}
\begin{itemize}
  \item arXiv:2602.01292, "Factorization algebras in quite a lot of generality" (2026-02-01)
  \item StackExchange cross-links via storage/mo-processed-gpu + math-processed-gpu
\end{itemize}
\subsection*{Synopsis}
Barwick introduces isolability structures and generalized sheaf theories that let factorization algebras live on a wide range of geometric objects, including arithmetic contexts. The key idea is that isolability encodes "distance" data via cographs, inducing a twofold symmetric monoidal structure on sheaf categories.
\begin{enumerate}
  \item defines isolability structures and the resulting twofold symmetric monoidal structure on sheaves;
  \item formulates factorization algebras within this framework and recovers known cases;
  \item constructs the Beilinson--Drinfeld Grassmannian as a factorization stack in this generality; and
  \item showcases applications to arithmetic and topological field theories.
\end{enumerate}
\subsection*{MathOverflow cues}
\begin{description}
  \item[MO 467210] Discussion on encoding "distant points" via cographs when defining factorization algebras.
  \item[MO 467288] Thread about extending Beilinson--Drinfeld Grassmannians to arithmetic contexts.
\end{description}
\subsection*{Math.SE anchors}
\begin{description}
  \item[Math.SE 502410] "Sheaf-theoretic input for isolability structures".
  \item[Math.SE 502557] "Factorization algebras beyond manifolds".
\end{description}
\subsection*{Encyclopedia outline}
\begin{enumerate}
  \item \textbf{Motivation.} Explain isolability structures and why they are minimal data (Math.SE 502410).
  \item \textbf{Twofold monoidal structure.} Describe the induced symmetric monoidal structures on sheaves (MO 467210).
  \item \textbf{Factorization algebra definition.} Present the generalized definition and recovery of classical cases (Math.SE 502557).
  \item \textbf{Beilinson--Drinfeld Grassmannian.} Outline the construction as a factorization stack (MO 467288).
  \item \textbf{Applications.} Sketch how arithmetic QFTs and topological examples plug into the framework.
\end{enumerate}
\subsection*{Action items}
\begin{itemize}
  \item Ingest the arXiv source for precise definitions/diagrams.
  \item Pull MO embeddings (467210, 467288) and Math.SE embeddings (502410, 502557).
  \item Draft the PlanetMath entry with a comparison table of isolability structures vs. classical geometries.
\end{itemize}
