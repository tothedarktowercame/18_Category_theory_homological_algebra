% PlanetMath proposal draft for arXiv:math/9808082
% Source corpora: arXiv math.CT eprints, storage/mo-processed-gpu, storage/math-processed-gpu
\section*{Iterated Monoidal Categories}
\subsection*{Source}
\begin{itemize}
  \item arXiv:math/9808082, "Iterated Monoidal Categories" (1998-08-18)
  \item StackExchange cross-links via storage/mo-processed-gpu + math-processed-gpu
\end{itemize}
\subsection*{Synopsis}
Balteanu, Fiedorowicz, Schwaenzl, and Vogt define iterated monoidal categories whose nerves model iterated loop spaces. Free iterated monoidal categories yield simplicial operads homotopy equivalent to little cubes operads, encompassing braided tensor categories as special cases.
\begin{enumerate}
  \item Introduces iterated monoidal categories and their relation to loop spaces.
  \item Shows group completion of nerves yields iterated loop spaces.
  \item Constructs finite simplicial operads equivalent to little cubes.
\end{enumerate}
\subsection*{MathOverflow cues}
\begin{description}
  \item[MO 481241] Iterated monoidal categories vs. iterated loop spaces.
  \item[MO 481268] Free iterated monoidal categories and operads.
\end{description}
\subsection*{Math.SE anchors}
\begin{description}
  \item[Math.SE 510861] "Examples of iterated monoidal categories".
  \item[Math.SE 510887] "Little cubes equivalences".
\end{description}
\subsection*{Encyclopedia outline}
\begin{enumerate}
  \item \textbf{Definitions.} Present iterated monoidal categories (Math.SE 510861).
  \item \textbf{Loop space connection.} Explain group-completion results (MO 481241).
  \item \textbf{Operads.} Describe free constructions and little cubes equivalence (MO 481268, Math.SE 510887).
  \item \textbf{Examples.} Mention braided tensor categories as special cases.
  \item \textbf{Applications.} Discuss H-space structures and quantum groups.
\end{enumerate}
\subsection*{Action items}
\begin{itemize}
  \item Import arXiv source for definitions and equivalences.
  \item Reference MO 481241/481268 and Math.SE 510861/510887.
  \item Highlight loop-space/operad connections for PlanetMath readers.
\end{itemize}
