% PlanetMath proposal draft for arXiv:2602.14708
% Source corpora: arXiv math.CT eprints, storage/mo-processed-gpu, storage/math-processed-gpu
\section*{Categorical hypergraph models for data fabrics}
\subsection*{Source}
\begin{itemize}
  \item arXiv:2602.14708, "A Unified Mathematical Framework for Distributed Data Fabrics: Categorical Hypergraph Models" (2026-02-16)
  \item StackExchange cross-links via storage/mo-processed-gpu + math-processed-gpu
\end{itemize}
\subsection*{Synopsis}
Shaska and Kotsireas propose a categorical hypergraph framework $\mathcal{F}=(D,M,G,T,P,A)$ for distributed data fabrics, modeling datasets, metadata, transformations, policies, analytics via a hypergraph on a distributed system. They embed the hypergraph in a modular tensor category, use braided monoidal structures, and analyze algorithmic complexity.
\begin{enumerate}
  \item defines the hypergraph-based data fabric structure;
  \item describes categorical semantics with datasets as objects and transformations as morphisms;
  \item embeds the hypergraph into a modular tensor category with braided monoidal structure; and
  \item studies NP-hard tasks, spectral methods, and CAP/CAL considerations.
\end{enumerate}
\subsection*{MathOverflow cues}
\begin{description}
  \item[MO 469830] Question on categorical models for data fabrics.
  \item[MO 469862] Discussion on modular tensor interpretations of data hypergraphs.
\end{description}
\subsection*{Math.SE anchors}
\begin{description}
  \item[Math.SE 506890] "Categorical hypergraphs".
  \item[Math.SE 506924] "Spectral methods for hypergraph alignment".
\end{description}
\subsection*{Encyclopedia outline}
\begin{enumerate}
  \item \textbf{Framework.} Describe $\mathcal{F}$ and distributed system model (Math.SE 506890).
  \item \textbf{Categorical semantics.} Explain objects/morphisms and braided structures (MO 469830).
  \item \textbf{Tensor category embedding.} Detail modular tensor perspective (MO 469862).
  \item \textbf{Complexity.} Summarize NP-hardness, spectral methods, CAP/CAL notes (Math.SE 506924).
  \item \textbf{PlanetMath links.} Connect to applied category theory and data science entries.
\end{enumerate}
\subsection*{Action items}
\begin{itemize}
  \item Pull the arXiv source for technical definitions.
  \item Fetch MO embeddings (469830, 469862) and Math.SE embeddings (506890, 506924).
  \item Draft the PlanetMath entry bridging categorical structures and data engineering.
\end{itemize}
