% PlanetMath proposal draft for arXiv:2602.14075
% Source corpora: arXiv math.CT eprints, storage/mo-processed-gpu, storage/math-processed-gpu
\section*{Similarity Algebra: Approximate algebraic and Lie structures}
\subsection*{Source}
\begin{itemize}
  \item arXiv:2602.14075, "Similarity Algebra: A Framework for Approximate Algebraic and Lie Structures with Collapse to Classical Algebra" (2026-02-15)
  \item StackExchange cross-links via storage/mo-processed-gpu + math-processed-gpu
\end{itemize}
\subsection*{Synopsis}
Ghojogh and Amanpour introduce similarity algebra, a hierarchy of approximate algebraic structures satisfying axioms up to tolerance $\varepsilon$. They prove collapse theorems showing $\varepsilon \to 0$ recovers classical structures, relate to fuzzy algebra, and define similarity categories for morphisms between approximate systems.
\begin{enumerate}
  \item Defines similarity groups/rings/Lie structures with quantitative tolerances.
  \item Proves collapse theorems under uniform error control.
  \item Situates similarity algebra relative to fuzzy algebra and categorifies morphisms.
\end{enumerate}
\subsection*{MathOverflow cues}
\begin{description}
  \item[MO 480751] Approximate associative/ Lie structures and collapse results.
  \item[MO 480785] Relationship between similarity and fuzzy algebra.
\end{description}
\subsection*{Math.SE anchors}
\begin{description}
  \item[Math.SE 510372] "Working with $\varepsilon$-similar rings".
  \item[Math.SE 510398] "Limit collapse of similarity Lie groups".
\end{description}
\subsection*{Encyclopedia outline}
\begin{enumerate}
  \item \textbf{Approximate structures.} Define similarity algebras (Math.SE 510372).
  \item \textbf{Collapse theorem.} Present convergence results (MO 480751, Math.SE 510398).
  \item \textbf{Comparison.} Detail relation to fuzzy algebra (MO 480785).
  \item \textbf{Categories.} Describe similarity algebra morphisms.
  \item \textbf{Applications.} Suggest modeling of noisy operations.
\end{enumerate}
\subsection*{Action items}
\begin{itemize}
  \item Import the arXiv source for definitions and proofs.
  \item Capture MO 480751/480785 and Math.SE 510372/510398 discussions.
  \item Highlight collapse and comparison results in PlanetMath.
\end{itemize}
