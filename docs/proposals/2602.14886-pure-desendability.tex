% PlanetMath proposal draft for arXiv:2602.14886
% Source corpora: arXiv math.CT eprints, storage/mo-processed-gpu, storage/math-processed-gpu
\section*{Conservative geometric functors via purity}
\subsection*{Source}
\begin{itemize}
  \item arXiv:2602.14886, "Conservative geometric functors via purity" (2026-02-16)
  \item StackExchange cross-links via storage/mo-processed-gpu + math-processed-gpu
\end{itemize}
\subsection*{Synopsis}
Castellana and Gómez introduce pure descendability in compactly generated triangulated categories to test when families of geometric functors are jointly conservative. They apply it to sequential limits of ring spectra.
\begin{enumerate}
  \item defines pure descendability and relates it to conservativity;
  \item applies the criterion to families of geometric functors;
  \item studies sequential limits of ring spectra; and
  \item provides examples/obstructions.
\end{enumerate}
\subsection*{MathOverflow cues}
\begin{description}
  \item[MO 469520] Question on purity and conservativity.
  \item[MO 469542] Thread on sequential limits and ring spectra.
\end{description}
\subsection*{Math.SE anchors}
\begin{description}
  \item[Math.SE 506540] "Pure descendability in triangulated categories".
  \item[Math.SE 506574] "Geometric functors and conservativity tests".
\end{description}
\subsection*{Encyclopedia outline}
\begin{enumerate}
  \item \textbf{Pure descendability.} Define and motivate (Math.SE 506540).
  \item \textbf{Conservativity criterion.} Explain the main theorem (MO 469520).
  \item \textbf{Sequential limits.} Discuss applications to ring spectra (MO 469542, Math.SE 506574).
  \item \textbf{Examples.} Provide cases where the criterion applies.
  \item \textbf{PlanetMath links.} Connect to entries on triangulated categories and spectral sequences.
\end{enumerate}
\subsection*{Action items}
\begin{itemize}
  \item Import the arXiv source for definitions/proofs.
  \item Pull MO embeddings (469520, 469542) and Math.SE embeddings (506540, 506574).
  \item Draft the PlanetMath entry summarizing the purity-based conservativity test.
\end{itemize}
