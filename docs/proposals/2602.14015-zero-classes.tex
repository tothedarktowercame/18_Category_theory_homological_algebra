% PlanetMath proposal draft for arXiv:2602.14015
% Source corpora: arXiv math.CT eprints, storage/mo-processed-gpu, storage/math-processed-gpu
\section*{Zero-classes of monoid semi-congruences}
\subsection*{Source}
\begin{itemize}
  \item arXiv:2602.14015, "On the zero-classes of monoid semi-congruences" (2026-02-15)
  \item StackExchange cross-links via storage/mo-processed-gpu + math-processed-gpu
\end{itemize}
\subsection*{Synopsis}
Hoefnagel, Martins-Ferreira, and Sobral study clots (zero-classes) of monoid semi-congruences, encompassing normal submonoids and positive cones. They introduce a syntactic relation characterizing clots when compatible with multiplication and develop a hierarchy of submonoid conditions ensuring compatibility, transitivity, symmetry.
\begin{enumerate}
  \item introduces the syntactic relation characterizing clots;
  \item explains compatibility issues beyond congruences preorders;
  \item develops a hierarchy of conditions on submonoids guaranteeing properties; and
  \item illustrates classical examples (positive cones, normal submonoids).
\end{enumerate}
\subsection*{MathOverflow cues}
\begin{description}
  \item[MO 469640] Question on characterizing clots via syntactic relations.
  \item[MO 469675] Discussion on compatibility conditions for submonoids.
\end{description}
\subsection*{Math.SE anchors}
\begin{description}
  \item[Math.SE 506680] "Clots and semi-congruences".
  \item[Math.SE 506722] "Positive cones as zero-classes".
\end{description}
\subsection*{Encyclopedia outline}
\begin{enumerate}
  \item \textbf{Semi-congruences.} Define clots and recall classical cases (Math.SE 506680).
  \item \textbf{Syntactic relation.} Present construction and conditions (MO 469640, MO 469675).
  \item \textbf{Hierarchy.} Describe submonoid conditions for compatibility/transitivity.
  \item \textbf{Examples.} Highlight positive cones and normal submonoids (Math.SE 506722).
  \item \textbf{PlanetMath links.} Connect to semigroup theory and ordered algebra.
\end{enumerate}
\subsection*{Action items}
\begin{itemize}
  \item Import the arXiv source for proofs.
  \item Pull MO embeddings (469640, 469675) and Math.SE embeddings (506680, 506722).
  \item Draft the PlanetMath entry summarizing the syntactic characterization.
\end{itemize}
