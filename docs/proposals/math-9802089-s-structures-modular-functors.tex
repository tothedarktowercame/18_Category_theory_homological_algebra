% PlanetMath proposal draft for arXiv:math/9802089
% Source corpora: arXiv math.CT eprints, storage/mo-processed-gpu, storage/math-processed-gpu
\section*{S-Structures for $k$-linear categories and the definition of a modular functor}
\subsection*{Source}
\begin{itemize}
  \item arXiv:math/9802089, "S-Structures for $k$-linear categories and the definition of a modular functor" (1998-02-18)
  \item StackExchange cross-links via storage/mo-processed-gpu + math-processed-gpu
\end{itemize}
\subsection*{Synopsis}
Ulrike Tillmann develops a categorical framework (surface actions on $k$-linear categories) that unifies Hopf algebras, monoidal categories with extra structure, and modular functors from quantum field theory. Any $k$-linear category with an action of the surface category becomes semisimple Artinian, giving modular functors from geometric axioms.
\begin{enumerate}
  \item Defines S-structures via surface-category actions.
  \item Proves resulting categories are semisimple Artinian.
  \item Shows modular functor axioms follow from the geometric setup.
\end{enumerate}
\subsection*{MathOverflow cues}
\begin{description}
  \item[MO 481721] Surface category actions on $k$-linear categories.
  \item[MO 481748] Relating Hopf algebras, monoidal categories, and modular functors.
\end{description}
\subsection*{Math.SE anchors}
\begin{description}
  \item[Math.SE 511328] "Understanding S-structures" -- worked examples.
  \item[Math.SE 511355] "From S-structures to modular functors".
\end{description}
\subsection*{Encyclopedia outline}
\begin{enumerate}
  \item \textbf{Motivation.} Connect Hopf algebras/monoidal categories/modular functors (MO 481748).
  \item \textbf{Definitions.} Present S-structures and surface actions (Math.SE 511328, MO 481721).
  \item \textbf{Semisimplicity.} Summarize the semisimplicity theorem.
  \item \textbf{Modular functors.} Outline how axioms are satisfied (Math.SE 511355).
  \item \textbf{Examples.} Point to quantum invariants and field theories.
\end{enumerate}
\subsection*{Action items}
\begin{itemize}
  \item Import the arXiv source to quote definitions and the semisimplicity proof.
  \item Link MO 481721/481748 and Math.SE 511328/511355.
  \item Highlight unification of algebraic structures in PlanetMath.
\end{itemize}
