% PlanetMath proposal draft for arXiv:math/9906082
% Source corpora: arXiv math.CT eprints, storage/mo-processed-gpu, storage/math-processed-gpu
\section*{On Morita's fundamental theorem for C*-algebras}
\subsection*{Source}
\begin{itemize}
  \item arXiv:math/9906082, "On Morita's Fundamental Theorem for C*-algebras" (1999-06-12)
  \item StackExchange cross-links via storage/mo-processed-gpu + math-processed-gpu
\end{itemize}
\subsection*{Synopsis}
Blecher solves a Morita equivalence problem for C*-algebras using operator space methods: C*-algebras are strongly Morita equivalent (Rieffel) iff their categories of left operator modules are equivalent via completely contractive functors, implemented by Haagerup tensoring with equivalence bimodules.
\begin{enumerate}
  \item States equivalence of strong Morita equivalence and operator module category equivalence.
  \item Shows implementing functors are Haagerup tensor products with Morita bimodules.
  \item Notes operator space framework necessity.
\end{enumerate}
\subsection*{MathOverflow cues}
\begin{description}
  \item[MO 484732] Operator module categories and Morita equivalence.
  \item[MO 484759] Haagerup tensor functors.
\end{description}
\subsection*{Math.SE anchors}
\begin{description}
  \item[Math.SE 514534] "Strong Morita equivalence via operator modules".
  \item[Math.SE 514560] "Haagerup tensor products implementing equivalences".
\end{description}
\subsection*{Encyclopedia outline}
\begin{enumerate}
  \item \textbf{Morita equivalence.} Review Rieffel's notion (Math.SE 514534).
  \item \textbf{Operator modules.} Describe categories and equivalence (MO 484732).
  \item \textbf{Haagerup tensor.} Explain implementation (MO 484759, Math.SE 514560).
  \item \textbf{Consequences.} Discuss implications for subcategories.
  \item \textbf{Examples.} Provide simple C*-algebras.
\end{enumerate}
\subsection*{Action items}
\begin{itemize}
  \item Import arXiv proof details and operator space arguments.
  \item Link MO 484732/484759 and Math.SE 514534/514560.
  \item Emphasize operator-space Morita characterization.
\end{itemize}
