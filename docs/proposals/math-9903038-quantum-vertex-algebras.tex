% PlanetMath proposal draft for arXiv:math/9903038
% Source corpora: arXiv math.CT eprints, storage/mo-processed-gpu, storage/math-processed-gpu
\section*{Quantum vertex algebras}
\subsection*{Source}
\begin{itemize}
  \item arXiv:math/9903038, "Quantum vertex algebras" (1999-03-07)
  \item StackExchange cross-links via storage/mo-processed-gpu + math-processed-gpu
\end{itemize}
\subsection*{Synopsis}
Borcherds treats vertex algebras categorically as ``singular commutative rings'' in a suitable category, simplifying constructions. Examples arise via analogues of twisted group rings built from bicharacters. Quantum vertex algebras become singular braided rings.
\begin{enumerate}
  \item Recasts vertex algebras as singular commutative rings.
  \item Provides twisted group ring analogues with bicharacters.
  \item Defines quantum vertex algebras as singular braided rings.
\end{enumerate}
\subsection*{MathOverflow cues}
\begin{description}
  \item[MO 483955] Singular commutative ring viewpoint on vertex algebras.
  \item[MO 483982] Building quantum vertex algebras via bicharacters.
\end{description}
\subsection*{Math.SE anchors}
\begin{description}
  \item[Math.SE 513725] "Categorical description of vertex algebras".
  \item[Math.SE 513751] "Singular braided rings".
\end{description}
\subsection*{Encyclopedia outline}
\begin{enumerate}
  \item \textbf{Categorical setup.} Present singular commutative ring picture (Math.SE 513725).
  \item \textbf{Construction techniques.} Explain bicharacter methods (MO 483982).
  \item \textbf{Quantum vertex algebras.} Describe singular braided ring definition (Math.SE 513751).
  \item \textbf{Examples.} Provide higher-dimensional analogues.
  \item \textbf{Applications.} Note ties to quantum groups.
\end{enumerate}
\subsection*{Action items}
\begin{itemize}
  \item Import the arXiv source for categorical machinery and examples.
  \item Link MO 483955/483982 and Math.SE 513725/513751.
  \item Highlight the simplified construction viewpoint.
\end{itemize}
