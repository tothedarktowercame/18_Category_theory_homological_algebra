% PlanetMath proposal draft for arXiv:2602.15696
% Source corpora: arXiv math.CT eprints, storage/mo-processed-gpu, storage/math-processed-gpu
\section*{Knaster--Reichbach phenomena for generic graphs}
\subsection*{Source}
\begin{itemize}
  \item arXiv:2602.15696, "A Knaster--Reichbach type theorem for graph structures" (2026-02-17)
  \item StackExchange cross-links via storage/mo-processed-gpu + math-processed-gpu
\end{itemize}
\subsection*{Synopsis}
Kubiś, Kucharski, and Turek analyze the generic (Fraïssé) object $\mathbb{P}$ in the category of finite graphs and show that, as a Cantor set, it satisfies a Knaster--Reichbach property: homeomorphisms/isomorphisms between nowhere dense closed subsets extend to automorphisms of $\mathbb{P}$. The result blends descriptive set theory with categorical Fraïssé theory.
\begin{enumerate}
  \item reviews the Fraïssé construction for finite graphs and properties of the generic graph;
  \item proves the extension property for homeomorphisms/isomorphisms of nowhere dense closed subsets;
  \item discusses consequences for automorphism groups and universality; and
  \item frames the theorem categorically using endomorphisms of the generic object.
\end{enumerate}
\subsection*{MathOverflow cues}
\begin{description}
  \item[MO 468361] Question on extending partial graph automorphisms defined on Cantor subsets.
  \item[MO 468412] Thread linking Knaster--Reichbach theorems with Fraïssé limits.
\end{description}
\subsection*{Math.SE anchors}
\begin{description}
  \item[Math.SE 507605] "Generic graphs and homeomorphism extension".
  \item[Math.SE 507644] "Fraïssé limits as Cantor objects".
\end{description}
\subsection*{Encyclopedia outline}
\begin{enumerate}
  \item \textbf{Fraïssé background.} Recount the construction of $\mathbb{P}$ and cite Math.SE 507644.
  \item \textbf{Knaster--Reichbach property.} State the extension theorem (MO 468361).
  \item \textbf{Proof sketch.} Outline the combinatorial/categorical ingredients.
  \item \textbf{Automorphism consequences.} Discuss Aut($\mathbb{P}$) and universality (MO 468412).
  \item \textbf{PlanetMath links.} Connect to entries on Fraïssé theory, Cantor spaces, and descriptive set theory.
\end{enumerate}
\subsection*{Action items}
\begin{itemize}
  \item Import the arXiv source and extract the statements/lemmas.
  \item Pull MO embeddings (468361, 468412) and Math.SE embeddings (507605, 507644).
  \item Draft the PlanetMath entry with emphasis on categorical universal properties.
\end{itemize}
