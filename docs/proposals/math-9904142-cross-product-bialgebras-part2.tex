% PlanetMath proposal draft for arXiv:math/9904142
% Source corpora: arXiv math.CT eprints, storage/mo-processed-gpu, storage/math-processed-gpu
\section*{Cross product bialgebras -- Part II}
\subsection*{Source}
\begin{itemize}
  \item arXiv:math/9904142, "Cross Product Bialgebras -- Part II" (1999-04-26)
  \item StackExchange cross-links via storage/mo-processed-gpu + math-processed-gpu
\end{itemize}
\subsection*{Synopsis}
Bespalov and Drabant develop the universal theory of cross product bialgebras with cocycles and dual cocycles, unifying bi-, smash, doublecross, bicross, double biproduct, and other constructions, while yielding new types of cross product bialgebras.
\begin{enumerate}
  \item Presents universal cocycle/dual-cocycle theory for cross product bialgebras.
  \item Provides equivalent (co-)modular co-cyclic formulation.
  \item Shows how existing constructions arise within the unified framework and produces new ones.
\end{enumerate}
\subsection*{MathOverflow cues}
\begin{description}
  \item[MO 483952] Cocycles in cross product bialgebras.
  \item[MO 483979] Modular formulations of cross products.
\end{description}
\subsection*{Math.SE anchors}
\begin{description}
  \item[Math.SE 513646] "Understanding cross product bialgebras".
  \item[Math.SE 513672] "Comparing bi-, smash-, and bicross products".
\end{description}
\subsection*{Encyclopedia outline}
\begin{enumerate}
  \item \textbf{Review Part I.} Recap cocycle-free theory (Math.SE 513646).
  \item \textbf{Universal cocycle theory.} Describe main construction (MO 483952).
  \item \textbf{Co-modular formulation.} Explain co-cyclic viewpoint (MO 483979).
  \item \textbf{Examples.} Show how classical constructions fit (Math.SE 513672).
  \item \textbf{New types.} Highlight novel bialgebra families.
\end{enumerate}
\subsection*{Action items}
\begin{itemize}
  \item Import Part II for cocycle and modular formulations.
  \item Link MO 483952/483979 and Math.SE 513646/513672.
  \item Emphasize unification of cross product frameworks.
\end{itemize}
