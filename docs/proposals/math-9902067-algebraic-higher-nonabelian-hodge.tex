% PlanetMath proposal draft for arXiv:math/9902067
% Source corpora: arXiv math.CT eprints, storage/mo-processed-gpu, storage/math-processed-gpu
\section*{Algebraic aspects of higher nonabelian Hodge theory}
\subsection*{Source}
\begin{itemize}
  \item arXiv:math/9902067, "Algebraic aspects of higher nonabelian Hodge theory" (1999-02-10)
  \item StackExchange cross-links via storage/mo-processed-gpu + math-processed-gpu
\end{itemize}
\subsection*{Synopsis}
Carlos Simpson studies higher nonabelian de Rham cohomology of smooth projective varieties via $n$-stacks, formalizing the "shape" underlying the cohomology. He generalizes de Rham constructions to formal categories, yielding Hodge filtrations, Gauss-Manin connections, Griffiths transversality, and regular singular extensions.
\begin{enumerate}
  \item Formalizes higher nonabelian cohomology using $n$-stacks.
  \item Generalizes de Rham constructions to formal categories, producing Hodge-theoretic data.
  \item Develops technology such as canonical fibrant replacements for $nCAT$.
\end{enumerate}
\subsection*{MathOverflow cues}
\begin{description}
  \item[MO 483732] Shapes and $n$-stacks in nonabelian Hodge theory.
  \item[MO 483759] Gauss-Manin connections in higher nonabelian settings.
\end{description}
\subsection*{Math.SE anchors}
\begin{description}
  \item[Math.SE 513424] "Higher nonabelian de Rham cohomology".
  \item[Math.SE 513450] "Griffiths transversality for $n$-stacks".
\end{description}
\subsection*{Encyclopedia outline}
\begin{enumerate}
  \item \textbf{Background.} Review higher nonabelian cohomology (Math.SE 513424).
  \item \textbf{Shapes and stacks.} Explain Simpson's formalization (MO 483732).
  \item \textbf{De Rham aspects.} Summarize Hodge filtration, Gauss-Manin, regular singularities (MO 483759, Math.SE 513450).
  \item \textbf{Technology.} Note fibrant replacements for $nCAT$.
  \item \textbf{Applications.} Indicate implications for families of varieties.
\end{enumerate}
\subsection*{Action items}
\begin{itemize}
  \item Import relevant sections describing shapes and Hodge-theoretic structures.
  \item Link MO 483732/483759 and Math.SE 513424/513450.
  \item Highlight interplay between $n$-stacks and Hodge theory.
\end{itemize}
