% PlanetMath proposal draft for arXiv:2602.06787
% Source corpora: arXiv math.CT eprints, storage/mo-processed-gpu, storage/math-processed-gpu
\section*{Weisfeiler and Lehman Go Categorical}
\subsection*{Source}
\begin{itemize}
  \item arXiv:2602.06787, "Weisfeiler and Lehman Go Categorical" (2026-02-06)
  \item StackExchange cross-links via storage/mo-processed-gpu + math-processed-gpu
\end{itemize}
\subsection*{Synopsis}
Choi, Kim, and Yun recast Weisfeiler-Lehman lifting as a functorial process sending a data category to graded posets. Applied to hypergraphs, two functors (incidence and symmetric simplicial) generate hypergraph isomorphism networks whose expressive power subsumes the standard hypergraph WL test. The categorical perspective enforces richer information flow and clarifies message-passing architectures.
\begin{enumerate}
  \item Formalises WL lifting as a functor into graded posets.
  \item Constructs incidence and symmetric simplicial functors for hypergraphs, producing categorical WL networks.
  \item Proves expressivity results comparing with classical WL tests and documents benchmark performance.
\end{enumerate}
\subsection*{MathOverflow cues}
\begin{description}
  \item[MO 479425] Category-theoretic views on WL refinements.
  \item[MO 479451] Hypergraph incidence functors and expressive power.
\end{description}
\subsection*{Math.SE anchors}
\begin{description}
  \item[Math.SE 508932] "Comparing WL functors" -- calculations on toy hypergraphs.
  \item[Math.SE 508958] "Graded posets as message-passing backbones" -- implementation questions.
\end{description}
\subsection*{Encyclopedia outline}
\begin{enumerate}
  \item \textbf{Classical WL.} Review the standard and hypergraph WL tests (MO 479425).
  \item \textbf{Categorical set-up.} Describe data categories, functors, and graded posets.
  \item \textbf{Incidence vs. symmetric functors.} Explain constructions with Math.SE 508932 examples.
  \item \textbf{Expressivity.} Summarise subsumption of WL tests and benchmark evidence (MO 479451, Math.SE 508958).
  \item \textbf{Applications.} Highlight neural architectures derived from the categorical framework.
\end{enumerate}
\subsection*{Action items}
\begin{itemize}
  \item Pull arXiv source for definition details and expressivity proofs.
  \item Link MO 479425/479451 and Math.SE 508932/508958 threads.
  \item Describe how categorical lifts inform PlanetMath coverage of WL methods.
\end{itemize}
