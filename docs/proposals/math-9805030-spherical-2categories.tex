% PlanetMath proposal draft for arXiv:math/9805030
% Source corpora: arXiv math.CT eprints, storage/mo-processed-gpu, storage/math-processed-gpu
\section*{Spherical 2-categories and 4-manifold invariants}
\subsection*{Source}
\begin{itemize}
  \item arXiv:math/9805030, "Spherical 2-categories and 4-manifold invariants" (1998-05-06)
  \item StackExchange cross-links via storage/mo-processed-gpu + math-processed-gpu
\end{itemize}
\subsection*{Synopsis}
Marco Mackaay defines nondegenerate finitely semisimple semi-strict spherical 2-categories and uses them to build state-sum invariants of triangulated 4-manifolds, proving invariance under Pachner moves. The construction generalizes Crane-Yetter/Crane-Frenkel invariants and includes examples from finite groups with 4-cocycles.
\begin{enumerate}
  \item Defines spherical 2-categories and required semisimplicity.
  \item Constructs 4-manifold state-sums and proves Pachner move invariance.
  \item Provides examples from finite groups/4-cocycles, relating to Dijkgraaf-Witten.
\end{enumerate}
\subsection*{MathOverflow cues}
\begin{description}
  \item[MO 481169] Spherical 2-categories and state-sums.
  \item[MO 481195] Connections to Crane-Yetter invariants.
\end{description}
\subsection*{Math.SE anchors}
\begin{description}
  \item[Math.SE 510786] "Constructing spherical 2-categories".
  \item[Math.SE 510812] "4D Pachner move checks".
\end{description}
\subsection*{Encyclopedia outline}
\begin{enumerate}
  \item \textbf{Definitions.} Outline spherical 2-categories (Math.SE 510786).
  \item \textbf{State-sum.} Detail invariant construction (MO 481169).
  \item \textbf{Pachner invariance.} Explain move-by-move proof (Math.SE 510812).
  \item \textbf{Examples.} Cover finite group 4-cocycle models (MO 481195).
  \item \textbf{Relations.} Compare to Crane-Yetter/Frenkel invariants.
\end{enumerate}
\subsection*{Action items}
\begin{itemize}
  \item Import arXiv text for definitions and invariance proofs.
  \item Gather MO 481169/481195 and Math.SE 510786/510812 references.
  \item Emphasize the finite group example linking to Dijkgraaf-Witten.
\end{itemize}
