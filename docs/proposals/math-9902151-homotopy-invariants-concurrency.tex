% PlanetMath proposal draft for arXiv:math/9902151
% Source corpora: arXiv math.CT eprints, storage/mo-processed-gpu, storage/math-processed-gpu
\section*{Homotopy invariants of higher dimensional categories and concurrency}
\subsection*{Source}
\begin{itemize}
  \item arXiv:math/9902151, "Homotopy invariants of higher dimensional categories and concurrency in computer science" (1999-02-26)
  \item StackExchange cross-links via storage/mo-processed-gpu + math-processed-gpu
\end{itemize}
\subsection*{Synopsis}
Philippe Gaucher models execution paths of parallel automata via strict globular $ω$-categories, defining globular homology and negative/positive corner homologies that capture looping, branching, and merging behaviors. Hurewicz morphisms relate globular homology to corner homologies, providing concurrency invariants.
\begin{enumerate}
  \item Defines globular homology and corner homologies for $ω$-categories.
  \item Constructs negative/positive Hurewicz morphisms.
  \item Interprets invariants in terms of branching/merging execution paths.
\end{enumerate}
\subsection*{MathOverflow cues}
\begin{description}
  \item[MO 483425] Globular vs. corner homology.
  \item[MO 483452] Hurewicz morphisms in concurrency.
\end{description}
\subsection*{Math.SE anchors}
\begin{description}
  \item[Math.SE 513115] "ω-categories modeling concurrency".
  \item[Math.SE 513141] "Computing globular homology".
\end{description}
\subsection*{Encyclopedia outline}
\begin{enumerate}
  \item \textbf{ω-categories and concurrency.} Introduce modeling viewpoint (Math.SE 513115).
  \item \textbf{Globular homology.} Describe definitions and meaning (Math.SE 513141).
  \item \textbf{Corner homologies.} Explain branching/merging invariants (MO 483425).
  \item \textbf{Hurewicz morphisms.} Summarize constructions (MO 483452).
  \item \textbf{Applications.} Mention concurrency examples.
\end{enumerate}
\subsection*{Action items}
\begin{itemize}
  \item Import the arXiv thesis for invariant definitions and proofs.
  \item Link MO 483425/483452 and Math.SE 513115/513141.
  \item Highlight interplay of higher categories and concurrency.
\end{itemize}
