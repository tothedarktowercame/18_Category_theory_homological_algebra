% PlanetMath proposal draft for arXiv:2602.12738
% Source corpora: arXiv math.CT eprints, storage/mo-processed-gpu, storage/math-processed-gpu
\section*{On the Kelly monoidal structure of $\Lambda$-sequences and unital operads}
\subsection*{Source}
\begin{itemize}
  \item arXiv:2602.12738, "On the Kelly monoidal structure of $\Lambda$-sequences and unital operads" (2026-02-13)
  \item StackExchange cross-links via storage/mo-processed-gpu + math-processed-gpu
\end{itemize}
\subsection*{Synopsis}
Fan and Zou study $\Lambda$-sequences (based finite sets with injections) under the Kelly monoidal product. They identify monoids and modules therein, prove that the forgetful functor to symmetric sequences is an isomorphism on right modules, extend compatible lower data to a universal normal oplax monoidal structure, and characterise unital operads as monoids in unital $\Lambda$-sequences.
\begin{enumerate}
  \item Analyses the Kelly product on $\Lambda$-sequences and right modules.
  \item Establishes a universal normal oplax monoidal structure extending the Kelly product.
  \item Characterises unital operads as monoids in unital $\Lambda$-sequences.
  \item Proves a closed monoidal localisation theorem.
\end{enumerate}
\subsection*{MathOverflow cues}
\begin{description}
  \item[MO 480055] Kelly products on $\Lambda$-sequences and modules.
  \item[MO 480086] Normal oplax structures and unital operads.
\end{description}
\subsection*{Math.SE anchors}
\begin{description}
  \item[Math.SE 509611] "From $\Lambda$-sequences to unital operads" -- sample computations.
  \item[Math.SE 509638] "Closed monoidal localisation for operadic modules".
\end{description}
\subsection*{Encyclopedia outline}
\begin{enumerate}
  \item \textbf{Preliminaries.} Define $\Lambda$-sequences and the Kelly product (MO 480055).
  \item \textbf{Modules.} Explain the isomorphism to symmetric sequence modules.
  \item \textbf{Oplax structure.} Describe the universal normal oplax data (Math.SE 509638, MO 480086).
  \item \textbf{Unital operads.} Characterise them as monoids and state the localisation theorem (Math.SE 509611).
  \item \textbf{Examples.} Provide explicit unital operads in enriched settings.
\end{enumerate}
\subsection*{Action items}
\begin{itemize}
  \item Import the arXiv source for proofs of the oplax structures and localisation theorem.
  \item Reference MO 480055/480086 and Math.SE 509611/509638.
  \item Highlight implications for enriched unital operads in PlanetMath.
\end{itemize}
