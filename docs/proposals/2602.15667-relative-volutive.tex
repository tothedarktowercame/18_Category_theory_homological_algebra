% PlanetMath proposal draft for arXiv:2602.15667
% Source corpora: arXiv math.CT eprints, storage/mo-processed-gpu, storage/math-processed-gpu
\section*{Relative and lax volutive categories}
\subsection*{Source}
\begin{itemize}
  \item arXiv:2602.15667, "Relative and lax volutive categories" (2026-02-17)
  \item StackExchange cross-links via storage/mo-processed-gpu + math-processed-gpu
\end{itemize}
\subsection*{Synopsis}
Lüders introduces relative volutive categories and their lax forms, showing that rigid symmetric monoidal categories admit volutive structures while closed symmetric monoidal categories admit lax ones. The paper develops equivalent presentations and gives examples from bornological vector spaces, star-rings, Hilbert spaces, and Morita 2-categories.
\begin{enumerate}
  \item defines relative/lax volutive structures and compares them with classical *-autonomous data;
  \item establishes equivalent formulations (e.g. involutive monads, dagger-like structures);
  \item analyzes examples coming from analytic and operator-algebraic settings; and
  \item explores Morita 2-categories inside fully closed symmetric monoidal 2-categories.
\end{enumerate}
\subsection*{MathOverflow cues}
\begin{description}
  \item[MO 468310] Query on extending volutive (*-autonomous) structures from rigid to merely closed monoidal categories.
  \item[MO 468402] Discussion on lax *-structures for bornological vector spaces.
\end{description}
\subsection*{Math.SE anchors}
\begin{description}
  \item[Math.SE 507450] "Examples of volutive structures in monoidal categories".
  \item[Math.SE 507518] "Morita 2-categories with dagger structures".
\end{description}
\subsection*{Encyclopedia outline}
\begin{enumerate}
  \item \textbf{Classical volutive categories.} Review *-autonomous/volutive structures and relate to Math.SE 507450.
  \item \textbf{Relative vs. lax versions.} Summarize the new definitions and cite MO 468310.
  \item \textbf{Equivalent presentations.} Compare involutive monads, daggers, and lax data.
  \item \textbf{Examples.} Detail the bornological, star-ring, Hilbert space, and Morita 2-category cases (MO 468402, Math.SE 507518).
  \item \textbf{PlanetMath links.} Plan cross-references to entries on *-categories, dagger categories, and Morita theory.
\end{enumerate}
\subsection*{Action items}
\begin{itemize}
  \item Import the arXiv source into futon6 and capture the equivalent characterizations.
  \item Pull MO embeddings (468310, 468402) and Math.SE embeddings (507450, 507518).
  \item Draft the PlanetMath article with a comparative table of relative vs. lax volutive data.
\end{itemize}
