% PlanetMath proposal draft for arXiv:math/9912059
% Source corpora: arXiv math.CT eprints, storage/mo-processed-gpu + math-processed-gpu
\section*{Combinatorics of branchings in higher dimensional automata}
\subsection*{Source}
\begin{itemize}
  \item arXiv:math/9912059, "Combinatorics of branchings in higher dimensional automata" (1999-12-08)
  \item StackExchange cross-links via storage/mo-processed-gpu + math-processed-gpu
\end{itemize}
\subsection*{Synopsis}
Gaucher analyzes branching areas of execution paths in $ω$-categories representing higher-dimensional automata, developing combinatorics for negative corner (branching) homology and reduced branching homology (branching complex modulo thin elements). He computes examples and invariance properties.
\begin{enumerate}
  \item Studies globular negative corner homology and reduced branching homology.
  \item Explains thin-element quotient and conjectures equivalence for freely generated automata.
  \item Computes branching homology for sample $ω$-categories and proves invariance results.
\end{enumerate}
\subsection*{MathOverflow cues}
\begin{description}
  \item[MO 485370] Branching homology of higher automata.
  \item[MO 485397] Thin elements and reduced branching homology.
\end{description}
\subsection*{Math.SE anchors}
\begin{description}
  \item[Math.SE 515271] "Negative corner homology".
  \item[Math.SE 515297] "Computing branching homology".
\end{description}
\subsection*{Encyclopedia outline}
\begin{enumerate}
  \item \textbf{$ω$-categories and automata.} Review modeling (Math.SE 515271).
  \item \textbf{Branching homology.} Summarize definitions (MO 485370).
  \item \textbf{Reduced homology.} Explain thin-element quotient (MO 485397).
  \item \textbf{Examples.} Provide computed cases (Math.SE 515297).
  \item \textbf{Invariance.} Describe invariance results.
\end{enumerate}
\subsection*{Action items}
\begin{itemize}
  \item Import arXiv definitions and example computations.
  \item Link MO 485370/485397 and Math.SE 515271/515297.
  \item Emphasize combinatorics of branching homology.
\end{itemize}
