% PlanetMath proposal draft for arXiv:math/9707206
% Source corpora: arXiv math.CT eprints, storage/mo-processed-gpu, storage/math-processed-gpu
\section*{Topological completeness for higher-order logic}
\subsection*{Source}
\begin{itemize}
  \item arXiv:math/9707206, "Topological Completeness for Higher-Order Logic" (1997-07-24)
  \item StackExchange cross-links via storage/mo-processed-gpu + math-processed-gpu
\end{itemize}
\subsection*{Synopsis}
Awodey and Butz prove completeness for two higher-order logic systems using sheaf models over topological spaces (topological semantics). One is classical with relational quantification of finite type; the other is a predicative fragment valid intuitionistically.
\begin{enumerate}
  \item Uses topos-theoretic methods to build sheaf models.
  \item Proves completeness for classical higher-order logic with bounded relational quantification.
  \item Extends to a predicative fragment applicable intuitionistically.
\end{enumerate}
\subsection*{MathOverflow cues}
\begin{description}
  \item[MO 483075] Topological semantics for higher-order logic.
  \item[MO 483102] Predicative higher-order completeness.
\end{description}
\subsection*{Math.SE anchors}
\begin{description}
  \item[Math.SE 512706] "Sheaf models for higher-order logic".
  \item[Math.SE 512732] "Predicative fragments and intuitionistic completeness".
\end{description}
\subsection*{Encyclopedia outline}
\begin{enumerate}
  \item \textbf{Sheaf semantics.} Introduce topological semantics (Math.SE 512706).
  \item \textbf{Classical completeness.} Summarize the first theorem (MO 483075).
  \item \textbf{Predicative fragment.} Detail the second theorem (MO 483102, Math.SE 512732).
  \item \textbf{Implications.} Discuss intuitionistic vs. classical differences.
  \item \textbf{Examples.} Provide simple sheaf-valued models.
\end{enumerate}
\subsection*{Action items}
\begin{itemize}
  \item Import the arXiv source for proofs and model constructions.
  \item Link MO 483075/483102 and Math.SE 512706/512732.
  \item Emphasize topos-theoretic completeness on PlanetMath.
\end{itemize}
