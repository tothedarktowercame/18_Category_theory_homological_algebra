% PlanetMath proposal draft for arXiv:2602.19422
% Source corpora: arXiv math.CT eprints, storage/mo-processed-gpu, storage/math-processed-gpu
\section*{Cohesive operads and shape theory}
\subsection*{Source}
\begin{itemize}
  \item arXiv:2602.19422, "Cohesive operads and shape theory" (2026-02-21)
  \item StackExchange cross-links via storage/mo-processed-gpu + math-processed-gpu
\end{itemize}
\subsection*{Synopsis}
This work blends cohesive $(\infty,1)$-topoi with operadic structures to encode shape-theoretic data.
\begin{enumerate}
  \item define cohesive operads via spans in cohesive topoi,
  \item prove a shape-invariance theorem for their algebras,
  \item compare with classical shape theory of Borsuk.
\end{enumerate}
\subsection*{MathOverflow cues}
\begin{description}
  \item[MO 467871] Geometric group theory spans that inform the cohesive symmetry discussion.
  \item[MO 466991] Set-theoretic/topological thread on descriptive spans used for shape invariants.
\end{description}
\subsection*{Math.SE anchors}
\begin{description}
  \item[Math.SE 486912] "What is a cohesive topos?"
  \item[Math.SE 495120] "Operads detecting shape invariants".
\end{description}
\subsection*{Encyclopedia outline}
Definitions, main theorems, comparison with classical shape theory, examples, references.
\subsection*{Action items}
Pull arXiv TeX, MO embeddings (467871, 466991), Math.SE embeddings (486912, 495120), then prepare PlanetMath entry.
