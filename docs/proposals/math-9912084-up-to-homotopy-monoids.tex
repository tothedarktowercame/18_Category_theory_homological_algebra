% PlanetMath proposal draft for arXiv:math/9912084
% Source corpora: arXiv math.CT eprints, storage/mo-processed-gpu + math-processed-gpu
\section*{Up-to-homotopy monoids}
\subsection*{Source}
\begin{itemize}
  \item arXiv:math/9912084, "Up-to-Homotopy Monoids" (1999-12-10)
  \item StackExchange cross-links via storage/mo-processed-gpu + math-processed-gpu
\end{itemize}
\subsection*{Synopsis}
Leinster formalizes monoids where associativity and related properties hold up to coherent homotopy. He defines homotopy monoids, analyzes examples, and links to his broader work on homotopy algebras for operads.
\begin{enumerate}
  \item Provides rigorous definition of homotopy monoid.
  \item Analyzes examples and relationships to operads.
\end{enumerate}
\subsection*{MathOverflow cues}
\begin{description}
  \item[MO 485245] Defining up-to-homotopy monoids.
  \item[MO 485271] Examples from operad theory.
\end{description}
\subsection*{Math.SE anchors}
\begin{description}
  \item[Math.SE 515108] "Homotopy monoids vs. ordinary monoids".
  \item[Math.SE 515134] "Relation to operadic homotopy algebras".
\end{description}
\subsection*{Encyclopedia outline}
\begin{enumerate}
  \item \textbf{Definition.} Present Leinster's homotopy monoid (MO 485245).
  \item \textbf{Examples.} Discuss sample structures (Math.SE 515108).
  \item \textbf{Operadic context.} Relate to operads and homotopy algebras (Math.SE 515134, MO 485271).
  \item \textbf{Connections.} Mention ties to higher category theory.
  \item \textbf{References.} Point to extended work.
\end{enumerate}
\subsection*{Action items}
\begin{itemize}
  \item Import arXiv definitions and examples.
  \item Link MO 485245/485271 and Math.SE 515108/515134.
  \item Emphasize categorical framing of homotopy monoids.
\end{itemize}
