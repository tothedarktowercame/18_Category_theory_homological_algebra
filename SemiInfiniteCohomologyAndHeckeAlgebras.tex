\documentclass[12pt]{article}
\usepackage{pmmeta}
\pmcanonicalname{SemiInfiniteCohomologyAndHeckeAlgebras}
\pmcreated{2026-03-01 00:20:00}
\pmmodified{2026-03-01 00:20:00}
\pmowner{codex}{0}
\pmmodifier{codex}{0}
\pmtitle{semi-infinite cohomology and Hecke algebras}
\pmrecord{13}{0}
\pmprivacy{1}
\pmauthor{codex}{0}
\pmtype{Encyclopedia Entry}
\pmcomment{PlanetMath entry based on arXiv:math/0004139}
\pmclassification{msc}{18G55}
\pmclassification{msc}{20G10}
\pmdefines{semi-infinite Hecke algebra}

\endmetadata

\usepackage{amssymb,amsmath,amsfonts,amsthm}
\begin{document}
Alexandr Sevostyanov constructed ``semi-infinite Hecke algebras'' attached to triples $(A,B,e)$ consisting of an associative algebra $A$, a subalgebra $B$, and a character $e:B\to k$ \cite{Sevostyanov}.  These algebras act naturally on semi-infinite cohomology groups, generalizing the action of classical Hecke operators on group cohomology.  For every left $A$-module $V$ and right $A$-module $W$, the semi-infinite Hecke algebra acts on the Borel-type cohomology and homology groups $H_B^*(V)$ and $H_*^B(W)$.  In particular, the $W$-algebra $W(\mathfrak{g})$ associated to a semisimple Lie algebra $\mathfrak{g}$ arises as a semi-infinite Hecke algebra, providing an explicit realization without the bosonization techniques of Feigin--Frenkel.

Sevostyanov defines the semi-infinite cohomology functor using the semi-regular bimodule and studies its compatibility with affine flag varieties.  The resulting algebras interpolate between Drinfeld’s double and classical Iwahori--Hecke algebras.

\paragraph{Related discussions.}  MathOverflow thread \#14764 surveys D-module approaches to Kac--Moody flag varieties and highlights the role of semi-infinite cohomology.  Question \#327495 explains geometric realizations of affine and nil-Hecke rings.  On Math.SE, question \#233323 describes the semi-infinite Bruhat order, while \#2130266 reviews the definition of $W$-algebras—both supply background for Sevostyanov’s constructions.

\begin{thebibliography}{9}
\bibitem{Sevostyanov} A. Sevostyanov, ``Semi-infinite cohomology and Hecke algebras,'' \emph{arXiv:math/0004139} (2000).
\end{thebibliography}
%%%%%
%%%%%
\end{document}
