\documentclass[12pt]{article}
\usepackage{pmmeta}
\pmcanonicalname{HomotopyLieRinehartPairs}
\pmcreated{2026-02-28 00:31:00}
\pmmodified{2026-02-28 00:31:00}
\pmowner{codex}{00000}
\pmmodifier{codex}{00000}
\pmtitle{homotopical algebra of Lie--Rinehart pairs}
\pmrecord{11}{999996}
\pmprivacy{1}
\pmauthor{codex}{00000}
\pmtype{Topic}
\pmclassification{msc}{18G55}
\pmdefines{Lie--Rinehart pair}
\pmdefines{SH Lie--Rinehart pair}
\pmdefines{Dwyer--Kan localization}

\endmetadata

\usepackage{amsmath, amssymb}

\begin{document}
Lie--Rinehart pairs $(A,M)$ generalize Lie algebroids to an algebraic context: the commutative dg algebra $A$ plays the role of functions and the $A$-module $M$ plays the role of derivations, equipped with a Lie bracket compatible with the $A$-structure. Pi\v{s}talo (arXiv:2601.02895) develops a homotopy theory for dg Lie--Rinehart pairs by comparing them with ``strong homotopy'' (SH) versions.

\medskip
\noindent\textbf{Model via Dwyer--Kan localization.} Consider the category of dg Lie--Rinehart pairs where $A$ is a semi-free cdga over a characteristic-$0$ field and $M$ is built as a cell complex of $A$-modules. Localizing this category at quasi-isomorphisms produces an $\infty$-category of interest. The paper shows that this $\infty$-category is equivalent to the Dwyer--Kan localization of the category of SH Lie--Rinehart pairs satisfying the same cofibrancy conditions. Moreover, the latter forms a category of fibrant objects, so it is amenable to Brown's abstract homotopy theory.

\medskip
\noindent\textbf{Cofibrations and BV-type resolutions.} SH Lie--Rinehart pairs admit a well-behaved class of cofibrations, which the author exploits to construct functorial factorizations and to prove lifting properties. These tools show uniqueness (up to homotopy) of Batalin--Vilkovisky type resolutions of dg Lie--Rinehart pairs, a key step when transporting structures from derived geometry to higher stacks.

\medskip
\noindent\textbf{Cartesian fibrations.} Restricting to dg pairs whose underlying cdga is of finite type, and imposing a different cofibrancy hypothesis, the functor $(A,M)\mapsto A$ becomes a Cartesian fibration with presentable fibers. This clarifies how Lie--Rinehart data vary over families of cdgas.

\medskip
Overall, the paper furnishes a homotopical toolkit for Lie--Rinehart geometry: Dwyer--Kan localization identifies the correct $\infty$-category, SH objects provide manageable fibrant replacements, and cofibration techniques allow explicit constructions inside derived algebraic geometry.

\bigskip
\noindent\textbf{Reference.}
\begin{itemize}
  \item D. Pi\v{s}talo, \emph{Homotopical algebra of Lie-Rinehart pairs}, arXiv:2601.02895 (2026).
\end{itemize}
\end{document}
