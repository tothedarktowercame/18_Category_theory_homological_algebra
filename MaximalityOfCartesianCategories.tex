\documentclass[12pt]{article}
\usepackage{pmmeta}
\pmcanonicalname{MaximalityOfCartesianCategories}
\pmcreated{2026-03-01 00:00:00}
\pmmodified{2026-03-01 00:00:00}
\pmowner{codex}{0}
\pmmodifier{codex}{0}
\pmtitle{maximality of cartesian categories}
\pmrecord{13}{0}
\pmprivacy{1}
\pmauthor{codex}{0}
\pmtype{Encyclopedia Entry}
\pmcomment{PlanetMath entry based on arXiv:math/9911059}
\pmclassification{msc}{18A05}
\pmclassification{msc}{18A40}
\pmdefines{maximal cartesian theory}

\endmetadata

\usepackage{amssymb,amsmath,amsfonts,amsthm}
\begin{document}
Kosta Do\v{s}en and Zoran Petri\'c showed in \cite{DosenPetric} that the usual equational axioms for finite products are \\emph{maximal}.  If one adds any further universally quantified equation (in the language with pairing, projections, and the terminal map) that is satisfied in every cartesian category, then the strengthened theory collapses: every model becomes equivalent to a preorder.  Their argument uses coherence for cartesian categories and the graphical calculus of binary trees, so it simultaneously explains why cartesian logic admits no non-trivial additional equational principles.

The theorem can be stated as follows.  Let $\mathcal{T}$ be the algebraic theory generated by the structural rules for a cartesian category.  Suppose $\sigma$ is another equation in the same language that holds in every cartesian category.  Then the theory $\mathcal{T}+\sigma$ proves that every morphism $A\to B$ factors through the terminal object, so the only models are preorders.  Hence $\mathcal{T}$ is maximal with respect to consistency.

Do\v{s}en and Petri\'c work in a proof-theoretic style, but the result can be reformulated categorically: if $\mathcal{C}$ is a cartesian category whose underlying preorder is not trivial, then its free completion under binary trees has no additional equations beyond those forced by the cartesian structure.  Equivalently, every non-degenerate cartesian category contains a free cartesian subcategory on one generator.

\paragraph{Context and related discussions.}  MathOverflow question \#354920 discusses why the equational theory of small categories cannot be algebraic; the present maximality result gives a complementary obstruction for the cartesian fragment.  MathOverflow thread \#382239 reviews the standard construction that turns a cartesian category into a monoidal one, highlighting the role of product coherence.  On the Math.SE side, question \#1505483 explains why familiar categories such as \textbf{Top} fail to be cartesian closed, while \#4318948 surveys additional axioms that force a cartesian category to degenerate to a preorder.  Taken together, these sources situate the maximality theorem among other rigidity phenomena for product structures.

\paragraph{Sketch of the proof.}  One interprets the language of cartesian categories by binary trees labelled with objects.  Every arrow expression $f : A\to B$ is represented by a tree built from projections and pairings.  If $f$ and $g$ denote distinct trees, the authors construct a cartesian category in which $f\neq g$ by evaluating the trees as endomorphisms of a free binary algebra.  This shows that no new equation can be satisfied by all cartesian categories unless it is derivable already.  When the extra equation is inconsistent with this free model, the only remaining models are preorders.

\paragraph{Consequences.}  The maximality phenomenon places sharp limits on attempts to axiomatize “cartesian quantum mechanics” or other non-classical logics with products: any extra equational principle that holds universally forces collapse.  It also clarifies the extent to which cartesian structure differs from monoidal structure, where additional coherence axioms (such as symmetry or braiding) do not cause degeneracy.

\begin{thebibliography}{9}
\bibitem{DosenPetric} K. Do\v{s}en and Z. Petri\'c, ``The maximality of cartesian categories,'' \emph{arXiv:math/9911059} (1999).
\end{thebibliography}
%%%%%
%%%%%
\end{document}
