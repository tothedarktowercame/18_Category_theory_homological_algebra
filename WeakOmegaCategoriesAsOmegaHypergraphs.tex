\documentclass[12pt]{article}
\usepackage{pmmeta}
\pmcanonicalname{WeakOmegaCategoriesAsOmegaHypergraphs}
\pmcreated{2026-03-01 00:15:00}
\pmmodified{2026-03-01 00:15:00}
\pmowner{codex}{0}
\pmmodifier{codex}{0}
\pmtitle{weak $\omega$-categories as $\omega$-hypergraphs}
\pmrecord{13}{0}
\pmprivacy{1}
\pmauthor{codex}{0}
\pmtype{Encyclopedia Entry}
\pmcomment{PlanetMath entry based on arXiv:math/0003137}
\pmclassification{msc}{18D05}
\pmclassification{msc}{18D35}
\pmdefines{$\omega$-hypergraph}

\endmetadata

\usepackage{amssymb,amsmath,amsfonts,amsthm}
\begin{document}
Hiroyuki Miyoshi and Toru Tsujishita proposed in \cite{MiyoshiTsujishita} a flexible notion of $\omega$-hypergraph that accommodates weak $\omega$-categories.  Unlike globular sets, an $n$-cell in an $\omega$-hypergraph may have many sources and targets, each equipped with a polarity that governs composition.  Directed $\omega$-hypergraphs are those in which every cell carries an orientation compatible with pasting.  The authors show how to endow directed $\omega$-hypergraphs with weak $\omega$-categorical structure by specifying $\omega$-identities, $\omega$-invertibility, and $\omega$-universality, thereby generalizing the Baez--Dolan definition.

\paragraph{Why hypergraphs?}  Hypergraphs capture the branching combinatorics of higher pasting diagrams better than globular trees.  This makes them suitable for describing coherence data in non-strict situations and for modeling multiple composition laws simultaneously.  MathOverflow discussion \#353952 analyzes when free strict $\omega$-categories remain free inside weak $(\infty,\infty)$-categories, while \#236453 explains the geometry of globes versus other shapes, setting the stage for hypergraph-based approaches.  Math.SE questions \#3190628 and \#358813 collect references for globular, simplicial, and opetopic treatments of higher categories, highlighting the diversity of models.

\paragraph{Main result.}  The authors construct a weak $\omega$-category structure on the $\omega$-hypergraph freely generated by an oriented tree.  They then prove that any directed $\omega$-hypergraph satisfying suitable regularity conditions can be completed to a weak $\omega$-category whose morphisms are described combinatorially.  This yields an explicit, polarity-based combinatorial model for higher categorical coherence.

\begin{thebibliography}{9}
\bibitem{MiyoshiTsujishita} H. Miyoshi and T. Tsujishita, ``Weak $\omega$-categories as $\omega$-hypergraphs,'' \emph{arXiv:math/0003137} (2000).
\end{thebibliography}
%%%%%
%%%%%
\end{document}
