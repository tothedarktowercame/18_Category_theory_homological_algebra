\documentclass[12pt]{article}
\usepackage{pmmeta}
\pmcanonicalname{StochasticTapeDiagrams}
\pmcreated{2026-02-28 00:29:00}
\pmmodified{2026-02-28 00:29:00}
\pmowner{codex}{00000}
\pmmodifier{codex}{00000}
\pmtitle{tapes as stochastic matrices of string diagrams}
\pmrecord{11}{999996}
\pmprivacy{1}
\pmauthor{codex}{00000}
\pmtype{Topic}
\pmclassification{msc}{18D99}
\pmdefines{tape diagram}
\pmdefines{stochastic string diagram}
\pmdefines{probabilistic Boolean circuit}

\endmetadata

\usepackage{amsmath, amssymb}

\begin{document}
Tape diagrams extend the usual string-diagram calculus to categories endowed with two monoidal products: a tensor $\otimes$ and a biproduct $\oplus$. They encode how a morphism simultaneously manipulates ``wires'' (the tensor) and distributes across ``branches'' (the biproduct). Bonchi--Cioffo (arXiv:2601.01472) show that, for Kleisli categories of the subdistribution monad, tape diagrams are equivalent to certain stochastic matrices assembled from string diagrams.

\medskip
\noindent\textbf{Two monoidal structures.} Given a category with a biproduct, one can regard morphisms as matrices whose entries are morphisms between tensor factors. The tape formalism turns this viewpoint into a graphical calculus where horizontal composition records matrix multiplication and vertical stacking records tensoring. When the category is the Kleisli category of the subdistribution monad, each matrix entry is a subdistribution, so tapes describe probabilistic behaviour.

\medskip
\noindent\textbf{Isomorphism with stochastic matrices.} The main theorem provides an explicit isomorphism between:
\begin{itemize}
  \item tapes built in the Kleisli category, and
  \item stochastic matrices whose entries are subdistributions of ordinary string diagrams.
\end{itemize}
Thus every tape is represented uniquely by a stochastic matrix, and conversely every such matrix defines a tape. The authors use this to transport algebraic reasoning (matrix equations) to the string-diagram world where diagrammatic reasoning tools apply.

\medskip
\noindent\textbf{Probabilistic Boolean circuits.} With the correspondence in place, the paper extracts a complete equational theory for probabilistic Boolean circuits: circuits built out of logical gates but evaluated in subdistributions. Soundness is inherited from the stochastic-matrix semantics, while completeness is proven diagrammatically. The result places probabilistic circuits inside the well-developed theory of monoidal categories with biproducts, making PlanetMath's coverage of string diagrams more useful for probabilistic computing.

\bigskip
\noindent\textbf{Reference.}
\begin{itemize}
  \item F. Bonchi, C.~J. Cioffo, \emph{Tapes as stochastic matrices of string diagrams}, arXiv:2601.01472 (2026).
\end{itemize}
\end{document}
